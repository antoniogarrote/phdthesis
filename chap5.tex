\chapter{Conclusiones y trabajo futuro}

A lo largo de los cap\'itulos de este documento se ha realizado una completa caracterizaci\'on de las APIs de datos sem\'anticos.\\
Para ello se ha tomado como punto de inicio la discusi\'on de los problemas actuales que se encuentran en el desarrollo web y que motivaron nuestro trabajo en primer lugar, as\'i como el repaso al estado actual de las distintas tecnolog\'ias y \'areas de investigaci\'on que se han tomado como base. A continuaci\'on se ha realizado un repaso de nuestra propuesta de soluci\'on desde distintos niveles: desde el punto de vista puramente formal hasta su implementaci\'on software y distintos ejemplos de aplicaci\'on de esta soluci\'on tecnol\'ogica a problemas reales de desarrollo web.\\
Por \'ultimo y a modo de conclusiones sobre el trabajo realizado, se pueden destacar los siguientes aspectos:

\begin{itemize}
  \item La web orientada a documentos se encuentra en pleno proceso de transformaci\'on en una red de datos y APIs que son consumidas por agentes software. El desarrollo de las plataformas m\'oviles y su ecosistema de aplicaciones, de servicios web sociales e incluso la incipiente \textit{Internet of Things} \cite{atzori2010internet} son los principales motores detr\'as de esta transformaci\'on.
  \item Esta nueva web de datos presenta importantes desaf\'ios desde el punto de vista del desarrollo web que todav\'ia no han sido resueltos, incluyendo problemas como la privacidad de los datos, el control de la informaci\'on, la interoperabilidad entre aplicaciones o la automatizaci\'on del acceso a los datos.
  \item Diversas tecnolog\'ias y areas de investigaci\'on desarrolladas en los \'ultimos a\~nos, tienen el potencial de solucionar diferentes aspectos relevantes a estos problemas. Dichas tencolog\'ias incluyen las arquitecturas \textit{REST}, ciertos desarrollos en el \'area de la Web Sem\'antica y la iniciativa \textit{Open Linked Data}, as\'i como la investigaci\'on en sistemas distribuidos.
  \item El concepto de API de Datos Sem\'anticos es un intento por combinar los avances en todos estos campos para ofrecer una interfaz pr\'actica, respetuosa con los conceptos arquitect\'onicos de la web, tal y como est\'an recogidos en los principios \textit{REST} y basada en est\'andares que facilite la interconexi\'on de datos entre agentes software y garantice la privcacidad y el control sobre sus datos a los usuarios.
  \item Desde el punto de vista puramente formal, las APIs de datos sem\'anticos y los agentes y servicios que interaccionan a trav\'es de ellos, se pueden modelar como un sistema distribuido descrito usando un c\'alculo de proceso similar al C\'alculo Pi con tipos y en el modelo de \textit{Espacios de Tripletes}. Esto permite razonar sobre la correcci\'on, requisitos y el dise\~no de las aplicaciones web que se van a implementar usando APIs sem\'anticas, adem\'as de abrir la puerta al tratamiento formal de problemas como la correcci\'on del dise\~no.
  \item Este modelo te\'orico se puede transformar en una arquitectura software para el desarrollo de APIs de datos basadas en los principios arquitecturales \textit{REST} y en las consideraciones fundamentales de la iniciativa \textit{Open Linked Data}, usando como modelo el est\'andar b\'asico de la Web Sem\'antica: \textit{RDF}.
  \item El uso de tecnolog\'ias sem\'anticas como \textit{RDF} no es incompatible con la utilizaci\'on de herramientas y tecnolog\'ias b\'asicas en el desarrollo web, con una gran base de usuarios y a\~nos de refinamiento en cuanto a rendimiento y estabilidad, como las bases de datos relacionales. Con el desarrollo de bibliotecas para la traducci\'on de consultas \textit{SPARQL} a \textit{SQL} y del modelo de datos \textit{RDF} al modelo de datos relacional, hemos visto como ambas tecnolog\'ias pueden hacerse compatibles.
  \item Con el desarrollo de una biblioteca \textit{JavaScript} que implementa un completo repositorio de tripletes con soporte para responder a consultas \textit{SPARQL} que se puede ejecutar dentro del navegador web, hemos mostrado como los agentes software de nuestro modelo formal pueden ser implementados dentro del cliente web por excelencia, el navegador web, adem\'as de demostrar la viabilidad del uso de tecnolog\'ias sem\'anticas fuera del lado servidor de la arquitectura web.
  \item A trav\'es del desarrollo de una aplicaci\'on agregadora de datos sociales que usa nuestra arquitectura de APIs de datos sem\'anticos y componentes software, hemos mostrado como dicho modelo puede usarse para solucionar algunos de los problemas de las redes sociales actuales, incluyendo la privacidad y autenticaci\'on, la integraci\'on de datos de diferentes servicios as\'i como la capacidad de dotar de control sobre sus propios datos a los usuarios de dichas aplicaciones.
  \item El accesso a datos sem\'anticos que garantiza nuestra arquitectura de APIs sem\'anticas, as\'i como la flexibilidad y expresividad del modelo de datos sem\'antico \textit{RDF} permite encontrar nuevas soluciones que simplifiquen problemas complejos como la construcci\'on de visualizaciones de datos interactivas, que incorporen datos provenientes d ediferentes servicios, tal y como hemos mostrado con el desarrollo de una biblioteca de visualizaci\'on de datos para el navegador, construida sobre nuestro repositorio \textit{RDF} \textit{JavaScript}.
\end{itemize}

Lejos de agotar la investigaci\'on en el \'area de la construcci\'on de interfaces de para datos sem\'anticos, el trabajo recogido en este documento abre la puerta a posibles desarrollos futuros que contin\'uen y expandan el trabajo aqu\'i realizado.\\
Como posibles l\'ineas de investigaci\'on abiertas podr\'iamos mencionar las siguientes:

\begin{itemize}
  \item La integraci\'on del trabajo realizado con tecnolog\'ias m\'oviles. En este documento se ha mostrado como los clientes web basados en navegadores web pueden ser asimilados a los agentes software de nuestro modelo formal, a trav\'es de bibliotecas software \textit{JavaScript} que les permiten interaccionar con los datos sem\'anticos. De una manera similar, los diferentes dispositivos m\'oviles se podr\'ian integrar como clientes de APIs sem\'anticas desarrollando los diferentes componentes software necesarios para que puedan interaccionar con los datos sem\'anticos recuperados desde dichas APIs. 
  \item Estos componentes sem\'anticos desarrollados para diferentes plataformas m\'oviles se podr\'ian utilizar no solo para comunicar aplicaciones con APIs sem\'anticas sino que se podr\'ian integrar como una \textit{capa sem\'antica} dentro de dicha plataforma, de forma tal que diferentes aplicaciones ejecut\'andose dentro del mismo terminal tengan acceso a los datos expuestos por otras aplicaciones, as\'i como los servicios de datos nativos del terminal usando est\'andares sem\'anticos. Esto supone una particularizaci\'on del modelo formal de interacci\'on entre agentes sem\'anticos que hemos descrito a un conjunto de procesos ejecut\'andose dentro del mismo dispositivo.
  \item El trabajo que se ha iniciado en la construcci\'on de componentes software para la implementaci\'on de nuestra arquitectura de APIs sem\'anticos podr\'ia expandirse para llevar a cabo el desarrollo de un verdadero \textit{framework} para el desarrollo de APIs sem\'anticas, que sea capaz de tomar como entrada la descripci\'on de una API descrita en \textit{RDF} y algunos par\'ametros de configuraci\'on y transformarla en una versi\'on ejecutable de dicha API lista para ser puesta en producci\'on.
  \item La arquitectura propuesta en este documento cubre el desarrollo de servicios de datos basados en el modelo tradicional \textit{REST} web, donde las peticiones son iniciadas siempre por los clientes. Nuevos desarrollos en tecnolog\'ias web, como los \textit{Web Sockets} \cite{hickson2011websocket}, est\'an transformando este modelo \textit{pull} en un modelo \textit{push} donde el servicio es capaz de iniciar la comunicaci\'on directamente con el cliente. Esta nuevas tecnolog\'ias web son compatible con nuestro modelo formal de APIs sem\'anticas, pero para poder integrarlas dentro de soluciones reales es necesario expandir nuestra propuesta arquitectural y resolver problemas t\'ecnicos como el env\'io de datos sem\'anticos a trav\'es de \textit{streams} de datos en tiempo real.
  \item Nuestro trabajo de adaptaci\'on de un sistema de almacenamiento de informaci\'on y consulta relacional a los est\'andares sem\'anticos para integrarlo dentro de nuestra arquitectura web, puede emularse dentro de otros sistemas de almacenamiento y recuperaci\'on de informaci\'on donde la integraci\'on de dichos datos con otros sistemas software resulta complicada. Un caso especialmente interesante donde se podr\'ia construir una capa de integraci\'on sem\'antica es el de los sistemas \textit{Big Data} como \textit{Hadoop} o \textit{Hive}.
  \item Un dominio de aplicaci\'on especialmente adecuado para estudiar la posible aplicaci\'on de APIs sem\'anticas lo constituye la configuraci\'on de sistemas de desarrollo, denominada \textit{dev ops} \cite{smith2011hype}. Diversas fuentes de datos, bibliotecas, m\'aquinas, repositorios de c\'odigo fuente, requisitos, etc se encuentran almacenados en sistemas aislados y desconectados mutuamente. La introducci\'on de tecnolog\'ias de \textit{Cloud Computing} en los escenarios de desarrollo han hecho el problema todav\'ia m\'as acuciante. El uso de tecnolog\'ias sem\'anticas y APIs de datos sem\'anticos, pueden ofrecer una soluci\'on para integrar todas estas fuentes de datos presentes en los entornos de desarrollo software actuales.

  \item Por \'ultimo, otro campo interesante para la aplicaci\'on de interfaces sem\'anticas entre sistemas es el de la denominada \textit{Internet of Things} donde la necesidad de intercambio de datos entre dispositivos con capacidades de computo muchas veces limitadas, podr\'ian permitir una propuesta est\'andar basada en nuestro modelo formal y arquitect\'onico para el intercambio de datos usando tecnolog\'ia sem\'antica.
\end{itemize}