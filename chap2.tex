\chapter{Estado del Arte}

Como se ha explicado en los cap\'itulos previos, el objetivo del presente trabajo es presentar una arquitectura y herramientas que permitan la construcci\'on de \textit{APIs} sem\'anticas para datos enlazados.\\
Para alcanzar este objetivo se ha hecho imprescindible un estudio pormenorizado del estado del arte en lo referente a la arquitectura de \textit{APIs} de datos \textit{REST}, modelos formales para la definici\'on de sistemas distribuidos, avances en el desarrollo de la Web Sem\'antica, los \'ultimos avances alcanzados dentro de la iniciativa \textit{Open Linked Data} as\'i como de las herramientas y aplicaciones disponibles para implementar todos estos conceptos.\\
Algunas de estas \'areas de investigaci\'on se encuentran directamente relacionadas, como por ejemplo, las iniciativas \textit{Open Linked Data} y Web Sem\'antica, mientras que otras solo se encuentran relacionadas de forma indirecta, como los trabajos en arquitecturas de servicios web \textit{REST} que han influido de una forma directa en el desarrollo de muchas propuestas \textit{Open Linked Data}, pero en algunos casos, como en la relaci\'on entre la teor\'ia de la descripci\'on formal de sistemas distribuidos, su relaci\'on con las otras \'areas de investigaci\'on es mucho m\'as difusa. Esto supone que ideas importantes para el presente trabajo aparecen en diferentes autores con aproximaciones distintas, aunque complementarias, y diferentes terminolog\'ias. En este cap\'itulo intentaremos establecer estas relaciones entre las diferentes bases conceptuales sobre las que se ha construido esta tesis, as\'i como definir una terminolog\'ia com\'un que se utilizar\'a en el resto de este documento.\\
En la primera secci\'on de este cap\'itulo, revisaremos los \'ultimos avances en arquitecturas de servicios web \textit{REST}, y los conceptos b\'asicos sobre Web Sem\'antica que se encuentran en el germen del desarrollo de esta tesis. A continuaci\'on, se examinaran los trabajos sobre sistemas distribuidos que se han utilizado para realizar una descripci\'on formal de las arquitecturas \textit{REST} y por \'ultimo, los principales desarrollos de la comunidad \textit{Open Linked Data} en los que se trata de establecer puentes entre las tecnolog\'ias de la Web Sem\'antica y las arquitecturas de servicios \textit{REST}.\\

\section{Arquitecturas de servicios web \textit{REST}}

En los \'ultimos diez a\~nos, la visi\'on de la Web como una plataforma sobre la que desplegar aplicaciones distribuidas altamente desacopladas y extensibles, compuestas por una multitud de servicios web de datos, construidos sobre la pila de protocolos web (\textit{HTTP}/\textit{TCP}/\textit{IP}) ha suscitado el inter\'es tanto de la industria como de la comunidad de investigadores y que han sido com\'unmente englobados bajo el ep\'igrafe de \textit{SOA} (\textit{Service Oriented Architectures} o \textit{Arquitecturas Orientadas a Servicios}) \cite{soa}.\\
Un servicio web de datos en una arquitectura \textit{SOA} se puede definir como unidades funcionales, at\'omicas, desacopladas y que pueden ser invocadas a trav\'es del protocolo \textit{HTTP}. En la concepci\'on \textit{SOA} de las arquitecturas de servicios web, los servicios son interoperables gracias al uso de meta-datos, informaci\'on adicional sobre el servicio que describe la funcionalidad ofrecida por el servicio as\'i como los mensajes y el protocolo para acceder a ella.\\
Cuando la funcionalidad que se pretende alcanzar va m\'as all\'a del simple acceso a un servicio web para recuperar datos, involucrando un gran n\'umero de servicios interaccionando entre ellos y con los clientes web, aparecen problemas complejos que debe ser resueltos: como la composici\'on de un conjunto de servicios web para conseguir una determinada funcionalidad, problema conocido como \textit{orquestaci\'on de servicios web}, o de c\'omo automatizar la interacci\'on de un conjunto de servicios web sin un punto central de control, problema conocido \textit{como coreograf\'ia de servicios web}. \cite{orchestration_choreography}\\
Otros problemas asociados incluyen el catalogado de servicios web, as\'i como el \textit{descubrimiento de capacidades de servicios web} \cite{hoschek2002web} as\'i como los problemas de seguridad y autenticaci\'on en el acceso a dichos servicios.\\
La aproximaci\'on m\'as com\'un inicialmente, al conjunto de problemas que presentan las \textit{arquitecturas orientadas a servicios} entre la industria vino dado por un conjunto de est\'andares conocidos de forma gen\'erica como arquitecturas \textit{SOA WS-*} \cite{viering2009lacking}. 
Los componentes fundamentales de las arquitecturas \textit{SOA WS-*} consisten en un grupo de est\'andares entre los que se pueden destacar:\\

\begin{itemize}
\item \textbf{\textit{SOAP} (\textit{Simple Object Access Protocol}) \cite{soap}:} Un protocolo y formato de serializaci\'on de datos basado en \textit{XML} y que puede ser utilizado sobre una capa de transporte como HTTP o SMTP para el intercambio de datos estructurados.

\item \textbf{\textit{WSDL} (\textit{Web Services Description Language}) \cite{wsdl}:} Un lenguaje para la descripci\'on de la funcionalidad ofrecida por un servicio web, incluyendo la localizaci\'on del servicio, la forma de acceso y la descripci\'on, usando \textit{SOAP}, de los par\'ametros para su invocaci\'on as\'i como el tipo de datos devuelto.

\item \textbf{\textit{UDDI} (\textit{Universal Description Discovery and Integration}) \cite{uddi}:} Una propuesta de cat\'alogo de servicios web, construido sobre \textit{SOAP} y \textit{WSDL}, que puede ser usado por clientes web para encontrar y e invocar servicios web con el fin de llevar a cabo una determinada funcionalidad.
\end{itemize}


Sobre estos bloques b\'asicos, la comunidad \textit{WS-*} ha propuesto especificaciones m\'as complejas, que intentan solucionar algunos de los problemas anteriormente mencionados, como el de la coreograf\'ia de servicios web a trav\'es de la propuesta de est\'andar \textit{WS-CDL} (\textit{Web Services Choreography Description Language}) \cite{wscdl} o problemas tales como la descripci\'on formal y automatizable de la interacci\'on entre un determinado n\'umero de clientes y servicios web, expresada como un flujo de trabajo, para llevar a cabo alg\'un tipo de funcionalidad compleja, con est\'andares como \textit{WS-BPEL} (\textit{Business Process Execution Language}) \cite{wsbpel}. El objetivo \'ultimo que se persegu\'ia con la especificaci\'on de este conjunto de est\'andares era la descripci\'on de una arquitectura de distribuidos que se pudiese usar para la construcci\'on de aplicaciones empresariales complejos, siendo el caso paradigm\'atico la construcci\'on de sistemas \textit{ERPs} (\textit{Enterprise Resource Planner}) \cite{jian2008design}, \cite{li2009research}, conocida usualmente como \textit{Enterprise Service Bus} (ESB) \cite{esb}, por analog\'ia con el \textit{bus} de datos en las arquitecturas hardware, con un enfoque m\'as ligero, extensible y f\'acil de mantener que alternativas anteriores basadas en otras propuestas tecnol\'ogicas como CORBA, gracias a las caracter\'isticas de ocultaci\'on de la informaci\'on, alta cohesi\'on y bajo acoplamiento que te\'oricamente ofrecen los servicios web.\\

A medida que las implementaciones pr\'acticas de arquitecturas \textit{SOA} siguiendo los est\'andares \textit{WS-*} empezaron a hacerse realidad a mediados de la d\'ecada de los 2000. Los resultados obtenidos dieron pie a posiciones cr\'iticas sobre la viabilidad de las arquitecturas \textit{SOA} tal y como son concebidas por este cuerpo de est\'andares. Los principales inconvenientes se\~nalados se pueden desglosar as\'i:

\begin{itemize}

\item \textbf{Complejidad innecesaria:} las especificaciones \textit{WS-*} introducen un conjunto nuevo de formatos, protocolos, meta-servicios, etc. que  a\~naden una capa de complejidad elevada encima de la relativamente sencilla capa de aplicaci\'on \textit{HTTP}. Esta capa de complejidad hace que los sistemas \textit{SOA WS-*} sean costosos de desarrollar y mantener, necesitando casi de forma obligatoria herramientas de desarrollo autom\'aticas capaces de generar todo el c\'odigo intermedio requerido para realizar invocaciones a servicios o exponer incluso la m\'as sencilla de las funcionalidades como un servicio web de acuerdo con los est\'andares. Esto evita que algunas de las promesas que las arquitecturas \textit{SOA} promet\'ian, como la facilidad de reemplazar implementaciones de servicios, sean dif\'iciles de obtener en la pr\'actica. 

\item \textbf{Problemas de eficiencia}: la complejidad introducida por los protocolos \textit{SOA WS-*} influye en las decisiones concretas de implementaci\'on de los servicios web siguiendo dichos est\'andares. Por ejemplo, \textit{SOAP} describe el formato del mensaje que debe ser intercambiado entre cliente y servicio como un documento \textit{XML} complejo con un gran impacto en el tama\~no final en \textit{bytes} de los datos intercambiados, que se ven considerablemente incrementados. Esto supone que los servicios web \textit{WS-*} pueden ser menos eficientes desde el punto de vista computacional respecto a otras alternativas de implementaci\'on m\'as ad-hoc, como los servicios sencillos \textit{RPC} (\textit{Remote Procedure Invocation}) \cite{xmlrpc}, a pesar de ser estos \'ultimos menos gen\'ericos y extensibles.

\item \textbf{Alto acoplamiento entre clientes y servicios:} a pesar de que los servicios \textit{WS-*} en s\'i no dependen unos de otros, los clientes intentando acceder a un servicio en particular dependen completamente del protocolo de acceso a ese servicio, lo que conlleva que  cualquier cambio en los par\'ametros o tipos de entrada del servicio implican un cambio obligatorio del c\'odigo del cliente.

\end{itemize}

Sin embargo, la principal cr\'itica recibida por las arquitecturas \textit{SOA WS-*} vino dada por lo que se percibi\'o como una falta de congruencia entre la capa adicional que es a\~nadida por los est\'andares \textit{WS-*} sobre la capa web \textit{HTTP} y los principios de dise\~no de dicha capa web.\\
Este conjunto de principios arquitecturales se encuentran ya definidos en documentos como los \textit{Axioms of Web Architecture} de Tim Berners-Lee \cite{berners1997axioms}, pero fueron sistematizados en la tesis de Roy Fielding, \textit{Architectural Styles and the Design of Network-based Software Architectures} \cite{fielding2000representational}. En dicha tesis la arquitectura de la Web, basada en el protocolo \textit{HTTP}, aparece descrita en el cap\'itulo usando el acr\'onimo \textit{REST} (\textit{Representational State Transfer}). Las caracter\'isticas de las arquitecturas \textit{REST} tal y como aparecen descritas en la tesis de Fielding se enumeran a continuaci\'on:

\begin{itemize}
\item \textbf{Orientada a recursos:} el bloque b\'asico de la arquitectura \textit{REST} es el recurso, entendido como una relaci\'on conceptual entre identificadores y un conjunto de datos u otros identificadores asociados a ese identificador. La sem\'antica d esta relaci\'on conceptual debe permanecer constante a lo largo del tiempo, aunque los datos asociados puedan variar.

\item \textbf{Ausencia de estado mutable:} en la arquitectura \textit{REST}, el servicio no almacena informaci\'on sobre el estado del cliente entre peticiones. Esto supone una importante ventaja ya que permite que los sistemas web sean altamente escalables implementando componentes como cach\'es o aumentando el n\'umero de servidores ofreciendo un recurso particular sin necesidad de coordinar entre ellos estado mutable.

\item \textbf{Uso de identificadores est\'andar:} basados en ele est\'andar \textit{URL}, que se puede utilizar para designar de forma \'unica cualquier recurso expuesto a trav\'es de la Web.

\item \textbf{M\'ultiples representaciones y negociaci\'on de contenido}: cuando un cliente web intenta acceder a un recurso a trav\'es usando el \textit{URL} que lo identifica, cliente y servidor deben decidir que representaci\'on particular del estado actual de ese recurso va a ser obtenida por el cliente. Esta representaci\'on es conocida en la arquitectura \textit{REST} como \textit{media type} y puede contener los bytes asociados a los datos del recurso, as\'i como meta-datos sobre el recurso e incluso meta-datos sobre los meta-datos.

\item \textbf{Interfaz uniforme \textit{HTTP}:} en la arquitectura \textit{REST} la interfaz para acceder a los recursos expuestos es uniforme y con una sem\'antica est\'andar bien definida. Esta interfaz se compone de los m\'etodos disponibles en el protocolo \textit{HTTP}: \textit{GET}, \textit{POST}, \textit{PUT}, \textit{PATCH, }\textit{DELETE}, \textit{HEAD}, \textit{OPTIONS} y \textit{TRACE} a los que se les asigna la sem\'antica de operaciones para obtener un recurso, crearlos, actualizarlo, destruirlo y obtener informaci\'on asociada con \'el respectivamente.

\end{itemize}


La tabla \ref{restvsws} muestra las principales diferencias entre las propuestas arquitecturales \textit{REST} y \textit{WS-*}.\\
C\'omo se puede observar, las principales diferencias vienen dadas por diferencias en principios conceptuales b\'asicos entre ambos enfoques. Por un lado, la concepci\'on del servicio web como una invocaci\'on de una operaci\'on remota arbitraria, en el caso de los servicios \textit{WS-*}, frente al servicio web como una petici\'on \textit{HTTP} est\'andar para recuperar o modificar un recurso usando de una forma estricta la sem\'antica del protocolo \textit{HTTP} en el caso de los servicios \textit{REST}. En comparaci\'on, los servicios \textit{WS-*} usan una sem\'antica variable para las invocaciones \textit{HTTP}, que hace obligatoria la introducci\'on de meta-datos y meta-servicios adicionales como \textit{UDDI}, aumentando de esta medida la complejidad de la implementaci\'on de soluciones \textit{WS-*}.\\
El resultado \'ultimo de la aparici\'on de la propuesta arquitectural \textit{REST} en el panorama de las arquitecturas orientadas a servicios fue que diferentes sistemas de computaci\'on empresarial basadas en soluciones \textit{WS-*} empezaron a ser reemplazados por soluciones construidas usando servicios web \textit{REST}. Sin embargo, algunos patrones de integraci\'on en aplicaciones empresariales no son f\'acilmente traducibles usando los principios arquitect\'onicos \textit{REST}, como por ejemplo, la distribuci\'on as\'incrona y confiable de mensajes o los escenarios composici\'on de servicios y descripci\'on de flujos de procesos de negocio \cite{pautasso2008restful}. Para poder implementar algunos de estos casos de uso usando servicios web \textit{REST}, es necesario recurrir a est\'andares adicionales como \textit{WADL} \cite{wadl} para la descripci\'on de los recursos. En estos casos las soluciones \textit{WS-*} siguen siendo usadas profusamente.\\
Al mismo tiempo, la propia pila de protocolos y est\'andares \textit{WS-*} ha venido incorporando ideas y conceptos \textit{REST} con el fin de ofrecer soluciones m\'as simples y escalables, por ejemplo la versi\'on 2.0 de \textit{WSDL} introduce soporte para describir servicios \textit{WS-*} implementados sobre servicios web \textit{REST} \cite{takase2008definition}.\\
Fuera del \'ambito de la computaci\'on empresarial, a mediados de la d\'ecada de los 2000 y coincidiendo con el auge de la llamada Web Social o Web 2.0 \cite{murugesan2007understanding}, aparece la necesidad entre los desarrolladores de aplicaciones web de ofrecer el acceso a los datos almacenados en esas aplicaciones a los usuarios para que pudieran ser utilizadas por aplicaciones cliente m\'oviles, aplicaciones software de terceros o dispositivos hardware, como c\'amaras fotogr\'aficas. Tambi\'en empez\'o a ser importante la integraci\'on de datos de diferentes aplicaciones para ofrecer aplicaciones, conocidas como mashups, que ofrec\'ian una nueva funcionalidad a partir de la composici\'on de las diferentes fuentes de datos.\\
Esto supon\'ia ofrecer interfaces de servicios web, englobadas en una \textit{API} de datos, que pudiesen ser consumidas por los clientes web y otros servicios. Tras algunos intentos iniciales de utilizar \textit{APIs} basadas en algunos est\'andares \textit{WS-*}, como \textit{SOAP}, la arquitectura \textit{REST} se ha impuesto como el marco conceptual elegido por la mayor\'ia de aplicaciones Web 2.0 para implementar sus \textit{APIs} de datos, con mayor o menor grado de fidelidad a los principios prescritos por dicha arquitectura.\\
La  implementaci\'on en populares \textit{frameworks} desarrollo web como \textit{Ruby on Rails} o \textit{Django} de estos principios tambi\'en ha contribuido decisivamente a su difusi\'on.
La caracter\'isticas comunes de este tipo de \textit{APIs} de datos en aplicaciones web se enumeran a continuaci\'on:

\begin{itemize}

\item \textbf{Soporte parcial para la interfaz uniforme:} donde s\'olo los m\'etodos \textit{GET} y \textit{POST}, los \'unicos disponibles en los navegadores web, son incluidos en la interfaz de acceso, desplaz\'andose las operaciones asociadas al \textit{REST}o de operaciones de la interfaz \textit{HTTP} a \textit{URLs} especiales de la aplicaci\'on.

\item \textbf{Ausencia de negociaci\'on de contenido:} Muchas \textit{APIs} de datos no soportan m\'as que una sola representaci\'on de los recursos expuestos. Cuando m\'as de una representaci\'on est\'a disponible, la soluci\'on m\'as usada es el recurso a utilizar diferentes \textit{URLs} para diferentes representaciones, rompi\'endose de esta manera la identidad \'unica del recurso.

\item \textbf{Uso de JSON como un formato universal de intercambio:} La importancia de \textit{JavaScript} como el lenguaje nativo del navegador web, as\'i como la sencillez del formato, su eficiencia y la presencia de buenas bibliotecas para serializar y deserializar datos en otros lenguajes de programaci\'on han hecho que \textit{JSON} (\textit{JavaScript Simple Object Notation}), se haya convertido en el formato por defecto en la mayor\'ia de \textit{APIs} de datos, desplazando a otras opciones m\'as populares en el \'ambito de los servicios web \textit{WS-*} como \textit{XML}.

\item \textbf{Uso limitado de las capacidades m\'as avanzadas del protocolo \textit{HTTP}:} Caracter\'isticas propias del protocolo \textit{HTTP} como el uso de las cabeceras de \textit{cach\'e} son ignoradas y otras veces, es frecuente encontrar usos err\'oneos de las mismas, como en el caso de los c\'odigos de retorno de las peticiones \textit{HTTP} o el uso de la cabecera de localizaci\'on del recurso creado, muchas veces ignorados en favor del retorno del estado en el cuerpo de los datos recuperados.

\item \textbf{Conflictos en la identidad de los recursos entre \textit{URIs} e identificadores:} De acuerdo con los principios \textit{REST}, un recurso web deber\'ia venir asociado a un o m\'as de un \textit{URI} estable. Sin embargo, la mayor\'ia de \textit{APIs} identifican los recursos no por un \textit{URI} sino por un identificador que se introduce en la representaci\'on del recurso. Esto es debido a factores anteriormente comentados, como el uso de diferentes \textit{URLs} para designar diferentes representaciones de un mismo recursos.

\item \textbf{Uso de mecanismos ad-hoc para establecer enlaces entre recursos:} El uso de identificadores arbitrarios y relativos s\'olo a una determinada \textit{API} de datos, hace imposible para el cliente usar un mecanismo est\'andar como una \textit{URI} para identificar y obtener los recursos relacionados con un determinado recurso. Para obtener estos recursos, el cliente debe computar la \textit{URL} desde la que el recurso relacionado estar\'a disponible a partir de la informaci\'on de estado del recurso actual de una forma espec\'ifica a la \textit{API} en la que est\'an englobados.

\item \textbf{Dif\'icil inter-operabilidad entre \textit{APIs}:} Las dificultades para establecer enlaces entre recursos de una misma \textit{API} anteriormente comentadas, se hacen patentes tambi\'en de forma todav\'ia m\'as evidente cuando se intentan enlazar recursos entre \textit{APIs} de diferentes proveedores, si ambos proveedores no est\'an haciendo uso de \textit{URIs} can\'onicos para identificar los recursos o si estos \textit{URIs} no pueden ser insertados de una forma simple en la representaci\'on de los recursos.

\end{itemize}

Una consecuencia directa de las caracter\'isticas anteriormente expuestas, es que la mayor\'ia de \textit{APIs} existentes hoy en d\'ia se encuentran aisladas unas de otras. Los datos expuestos por los diferentes servicios y aplicaciones web a trav\'es de \textit{APIs} incompatibles y que no usan mecanismos est\'andar para identificar y describir los datos que ofrecen, y con ellos los usuarios de esos datos, se encuentran encerrados en silos de informaci\'on o \textit{walled gardens} \cite{halpin2008beyond} constituyendo islas de datos desde las que no se pueden establecer conexiones con otros servicios de datos. Esta situaci\'on es similar a la de la pl\'etora de redes locales aisladas en la \'epoca anterior al protocolo \textit{IP} de red y \textit{HTTP} de aplicaci\'on que almacenaban repositorios de documentos aislados unos de otros, en formatos incompatibles y sin ofrecer la posibilidad de establecer enlaces entre documentos situados en dos subredes diferentes.\\
La b\'usqueda de mecanismos que permitan el enlazado y la conexi\'on de \textit{APIs} ofrecidas por diferentes servicios, de tal forma que se pueda usar un mecanismo com\'un para la descripci\'on de los datos y la identificaci\'on de los recursos propios y remotos, facilit\'andose de esta manera el descubrimiento y la automatizaci\'on del consumo de servicios de datos ha sido el objeto de intensa investigaci\'on durante los \'ultimos a\~nos.
Algunas de las principales l\'ineas de investigaci\'on que se encuentran actualmente abiertas y que son relevantes para la presente tesis, se analizan a continuaci\'on.

\subsection{Descripci\'on de \textit{APIs} \textit{REST}}

Uno de los principales objetivos de dise\~no \textit{REST} es el de que los servicios web construidos deben ser auto-descriptivos \cite{berners1997axioms}. Esto quiere decir que un cliente que desee consumir un recurso a trav\'es de un servicio web \textit{REST}, deber\'ia obtener de forma automatizable, toda la informaci\'on necesaria para acceder al recurso o manipularlo, obtener diferentes representaciones del mismo a trav\'es del propio servicio usando meta-datos asociados con el recurso.\\
Debido a la simplicidad caracter\'istica de los servicios web \textit{REST}, as\'i como por el uso de convenciones como la sem\'antica asociada por el protocolo \textit{HTTP} a las operaciones del protocolo, el  \'unico mecanismo est\'andar disponible en la arquitectura \textit{HTTP} para describir las capacidades de un determinado servicio e intercambiar otros meta-datos sobe el servicio es el conjunto de cabeceras \textit{HTTP} intercambiadas por clientes y servicios, as\'i como los c\'odigos de respuesta establecidos en el protocolo.\\
Estas cabeceras constituyen un mecanismo que puede ser usado de forma eficaz para descubrir informaci\'on sobre el servicio, como por ejemplo, en el mecanismo de negociaci\'on de contenido usando la cabecera de \textit{media type} asociada a un recurso. Adem\'as las cabeceras \textit{HTTP} son tambi\'en un mecanismo extensible, ya que nuevas cabeceras no recogidas en el est\'andar \textit{HTTP} puede ser incluidas como parte de una petici\'on de un cliente o en la respuesta proveniente de un servicio.\\
Sin embargo, el uso de cabeceras por s\'i s\'olo no resuelve completamente el problema de la construcci\'on de servicios auto-descriptivos, ya que constituye un mecanismo para el intercambio de meta-datos entre cliente y servicio pero no menciona como han de ser estos meta-datos, ni su formato, m\'as all\'a de los especificado por las diferentes versiones del protocolo \textit{HTTP}.
El problema se puede extrapolar a la descripci\'on de una \textit{API} completa, as\'i como la relaci\'on entre los recursos de esa \textit{API}.\\
En la actualidad este es un problema abierto. Actualmente la mayor\'ia de descripciones de \textit{APIs} \textit{REST} se realizan en texto plano, como documentaci\'on asociada al servicio que los desarrolladores del software que acceder\'a a dichos servicios deben interpretar y transformar en la l\'ogica de sus programas. Esta situaci\'on contrasta con los complejos mecanismos de descripci\'on de servicios web \textit{WS-*} que hacen muy recomendable el uso de herramientas autom\'aticas que generen el c\'odigo necesario para realizar las invocaciones pertinentes a los servicios que se desea consumir. \\
La forma de encontrar formas eficientes de describir servicios \textit{REST}, que puedan ser consumidas de forma autom\'atica por agentes software, sin incurrir en la complejidad asociada a los mecanismos de  descripci\'on de servicios \textit{WS-*} es todav\'ia un campo activo de investigaci\'on tanto a nivel acad\'emico como industrial.\\
Una primer tipo de soluci\'on a la descripci\'on de servicios web \textit{REST} ha consistido en la proposici\'on de diferentes lenguajes para la descripci\'on de recursos, como \textit{WADL} (\textit{Web Application Description Language}) \cite{wadl}, que propone una serializaci\'on \textit{XML} de la descripci\'on de los recursos de una \textit{API}, as\'i como de las operaciones, par\'ametros de entrada y c\'odigos de estado que ser\'an devueltos por los diferentes servicios de la \textit{API}. Esta descripci\'on pod\'ia ser expuesta como un recurso m\'as de la propia \textit{API} al que los clientes deben acceder para tener acceso al \textit{REST}o de recursos disponibles.\\
WADL supone una simplificaci\'on de mecanismos similares propuestos en el mundo de los servicios web \textit{WS-*} com \textit{WSDL}, y no ha conseguido el apoyo mayoritario de los desarrolladores de \textit{APIs} web \textit{REST}. El propio \textit{WSDL} en su versi\'on 2.0, permite la descripci\'on de servicios web \textit{REST}, como ya hemos mencionados, siendo esta versi\'on ligera de \textit{WSDL} mucho m\'as usado en la actualidad. Adem\'as de \textit{WADL}, otra gran cantidad de lenguajes de descripci\'on de servicios \textit{REST}, m\'as o menos ad-hoc y centrados en dominios concretos de aplicaci\'on han sido propuestos en la literatura sobre la materia. Entre ellos podemos citar \textit{RSWS} y \textit{WDL} \cite{lanthaler2010towards}.\\
Frente a los esfuerzos por especificar un lenguaje, m\'as simple que \textit{WSDL}, para la descripci\'on de servicios \textit{REST} que pueda ser usado de forma gen\'erica por los constructores de \textit{APIs} basadas en servicios \textit{REST}, una parte de la comunidad acad\'emica y de forma casi un\'anime la industria, ha argumentado que no existe la necesidad de un lenguaje de descripci\'on en absoluto.\\
La justificaci\'on a esta postura ha venido dada por la importante cantidad de informaci\'on acerca de la sem\'antica de los servicios que se puede obtener a partir de las convenciones propias de las arquitecturas \textit{REST}, como el uso de la interfaz uniforme \textit{HTTP} \cite{lanthaler2010towards}.\\

\subsection{Descubrimiento de servicios web \textit{REST}}

Un aspecto de gran importancia a la hora de implementar arquitecturas de servicios es la capacidad de descubrir autom\'aticamente los servicios necesarios para llevar a cabo una determinada funcionalidad. La principal opci\'on para el descubrimiento de servicios en el \'ambito de las arquitecturas de servicios web \textit{WS-*} la constituye \textit{UDDI} un registro centralizado de servicios, pensado para ser usado como un servicio de \textit{p\'aginas verdes} incluyendo detalles t\'ecnicos y de contacto sobre un determinado servicio.\\

\textit{UDDI} no ha encontrado una gran aceptaci\'on entre la industria, especialmente por su naturaleza centralizada as\'i como por su complejidad. La comunidad \textit{REST} ha intentando buscar en los \'ultimos a\~nos alternativas a sistemas como \textit{UDDI} que puedan usarse para descubrir servicios \textit{REST} sin incurrir en el complejo dise\~no de esta tecnolog\'ia. Algunas nuevas propuestas dentro de la comunidad de servicios web \textit{WS-*}, como \textit{WSIL} (\textit{Web Services Inspection Language} \cite{wsil} han intentado dar respuesta a las limitaciones de \textit{UDDI}, como por ejemplo, la arquitectura centralizada del servicio, pero sin conseguir a pesar de ello una importante aceptaci\'on.\\
Una de las soluciones exploradas ha sido el uso de un mecanismo fundamental en el dise\~no de la web como el sistema \textit{DNS} (\textit{Domain Name Service}) para aplicarlo al descubrimiento de servicios web \textit{REST} \cite{jara2012light} a trav\'es del uso de \textit{DNS-SD} una extensi\'on del est\'andar \textit{DNS} para el registro de servicios de un determinado tipo en un determinado dominio que pueden ser descubiertos por los clientes \textit{DNS}. La descripci\'on de los meta-datos del servicio se pueden incluir en un registro DNS de tipo TXT o SRV y conducir a trav\'es de una petici\'on \textit{HTTP} al servicio o a una descripci\'on del servicio usando alguno de los lenguajes de descripci\'on de servicios mencionados anteriormente. Este sistema supone algunas importantes ventajas sobre tecnolog\'ias como \textit{UDDI}: re-usa una tecnolog\'ia existente y aceptada por la industria, no es un sistema centralizado con lo que consigue una gran escalabilidad y es un sistema din\'amico donde nuevos servicios pueden registrarse y desregistrarse a medida que la computaci\'on evoluciona.\\
A pesar de las ventajas que una tecnolog\'ia como \textit{DNS} parece aportar como posible soluci\'on para el descubrimiento de servicios web, otra parte de la comunidad \textit{REST} ha intentado construir mecanismos para el descubrimiento de servicios sin hacer uso de elementos externos a la pila de tecnolog\'ias que conforman la web. Un ejemplo se encuentra en \cite{verborgh2011description}, donde se describe un algoritmo para realizar el descubrimiento de servicios web \textit{REST} con los que llevar a cabo una determinada tarea, partiendo de un enlace web inicial y utilizando para ello, la cabecera \textit{OPTIONS} del protocolo \textit{HTTP} como la forma con la que obtener los meta-datos asociados al servicio y el \textit{REST} o de cabeceras del protocolo \textit{HTTP} para negociar una posible representaci\'on o las capacidades asociadas al servicio.
Un modelo alternativo para el descubrimiento de servicios \textit{REST} descentralizado y basado puramente en tecnolog\'ias web que cumple con los est\'andares \textit{REST}, viene descrito en \cite{umbrich2009discovering}. En dicho modelo se distinguen tres capas principales: el conjunto de est\'andares web usados en el mecanismo de descubrimiento como \textit{URIs} y el protocolo \textit{HTTP}, una capa de referencia en la que se describe como, a partir de un determinado \textit{URI} se puede obtener la descripci\'on asociada a ese servicio y por \'ultimo la capa de descripci\'on, donde se investiga que vocabulario es el adecuado para especificar la descripci\'on del recurso.\\
Algunas de las propuestas de est\'andar que se han realizado para llevar a cabo la funcionalidad propia de la capa de referencia incluyen \textit{XLink} (\textit{XML Linking Language})  y \textit{LRDD} (\textit{Links based Resource Descriptor Discovery}) \cite{lrdd}. \textit{LRDD} describe mecanismos alternativos para asociar descriptores con los recursos descritos: uso de una etiqueta \textit{LINK} en la representaci\'on \textit{HTML} y \textit{Atom} de un recurso que enlazar\'ia al descriptor de ese recurso, el uso de una cabecera \textit{HTTP} denominada \textit{Link} que devolver\'ia el enlace a la descripci\'on del recurso cuando la \textit{URI} del recurso es desreferenciada usando una petici\'on \textit{HTTP} \textit{GET} o \textit{HEAD} y, por \'ultimo, el uso de un fichero \textit{.well-known} disponible en una localizaci\'on est\'andar para un dominio espec\'ifico y que incluir\'ia enlaces para las descripciones de los recursos disponibles en ese recurso.\\

\subsection{Flujos de trabajos y \textit{mashups} de servicios \textit{REST}}

En apartados anteriores hemos discutido los avances realizados en la descripci\'on de meta-informaci\'on acerca de recursos web, as\'i como de mecanismos que permitan el descubrimiento de dichos servicios. \\
La combinaci\'on de ambas l\'ineas de investigaci\'on abre la puerta a la automatizaci\'on de interacciones complejas entre agentes y servicios web, en lo que se conoce como ejecuci\'on de flujos de trabajo. Para llevar a cabo uno de estos flujos, los agentes deben descubrir, seleccionar y consumir servicios en un determinado orden de forma aut\'onoma siguiendo la gu\'ia ofrecida por una especificaci\'on de alto nivel donde se expone la l\'ogica de negocio que se quiere obtener. El proceso de transformar de una forma automatizada esta descripci\'on de la funcionalidad de negocio que se desea conseguir en un conjunto de interacciones entre servicios y agentes, se conoce como \textit{orquestaci\'on de servicios web} \cite{orchestration_choreography}.\\
Otro problema relacionado, aunque m\'as sencillo en su planteamiento es el la descripci\'on de interacciones entre servicios web a trav\'es de primitivas b\'asicas, como la composici\'on de servicios, que ser\'a luego ejecutada de una forma autom\'atica por parte de agentes software. Esta versi\'on del problema se conoce como \textit{coreograf\'ia de servicios web} \cite{orchestration_choreography}.\\
En el mundo de los servicios web \textit{WS-*}, dos est\'andares cubren ambas \'areas. \textit{WS-BPEL} (\textit{Business Processes Execution Language}) \cite{wsbpel} permite describir escenarios de orquestaci\'on de servicios web para que puedan ser luego ejecutados autom\'aticamente por un motor de ejecuci\'on de orquestaci\'on de servicios web. Por su parte \textit{WS-CDL} (\textit{Choreography Description Language}) \cite{wscdl} permite describir las operaciones b\'asicas que conforman la coreograf\'ia de servicios web para que puedan ser ejecutadas de una forma aut\'onoma por un agente software.\\
Un problema que ha sido abordado por diferentes autores es el de la integraci\'on de servicios web \textit{REST} con los est\'andares de coreograf\'ia y orquestaci\'on dise\~nados para los servicios web \textit{WS-*} \cite{zur2005developing}.\\
Un primer paso para lograr esta integraci\'on ha venido dada por el soporte para servicios \textit{REST} en las \'ultimas versiones de \textit{WSDL} \cite{takase2008definition}, que es un componente esencial para las especificaciones tanto de coreograf\'ia \textit{WS-CDL}, como de orquestaci\'on \textit{WS-BPEL}.\\
Desde un punto de vista complementario tambi\'en se ha intentado modelar las primitivas b\'asicas de orquestaci\'on de un lenguaje como \textit{BPEL} a las primitivas b\'asicas de la interfaz uniforme \textit{HTTP} que caracteriza a los servicios web \textit{REST} \cite{pautasso2010restful}.\\
Por \'ultimo tambi\'en se han dise\~nado diferentes lenguajes de orquestaci\'on y coreograf\'ia espec\'ificamente pensados para ser utilizados en un entorno de servicios web \textit{REST}, entre los que pod\'iamos destacar: \textit{SWAP}, \textit{Wf-XML}, \textit{AWSP} y \textit{ASAP} \cite{zur2005developing}.\\
El inter\'es por integrar servicios web \textit{REST} en las soluciones de automatizaci\'on de flujos de trabajo \textit{WS-*}, ya sea desde la perspectiva de la orquestaci\'on o la coreograf\'ia, ha venido dada en la mayor\'ia de los casos por miembros de la industria que han invertido tecnol\'ogicamente en soluciones \textit{WS-*} y desean integrar el, cada vez m\'as rico, ecosistema de servicios web \textit{REST} dentro de su infraestructura. Sin embargo, dentro de la comunidad \textit{REST} y m\'as concretamente dentro de la comunidad de desarrollo web, existen problemas relacionados con la composici\'on de servicios web que han sido objetos de investigaci\'on tanto por la industria como en entornos acad\'emicos.\\
El caso m\'as paradigm\'atico es el de la construcci\'on de \textit{mashups} web \cite{mashups}. Se trata de un caso particular de orquestaci\'on de servicios, donde diferentes servicios \textit{REST} deben ser compuestos para obtener una funcionalidad web que ser\'a expuesta a un usuario. En la mayor\'ia de los casos el consumo y la integraci\'on de los recursos expuestos por los servicios \textit{REST} ser\'a autom\'atica, pero el dise\~no de la forma en que estos recursos servicios se agregar\'an estar\'a dirigida de una forma manual e interactiva por parte del usuario final a trav\'es de alg\'un tipo de interfaz web.
La mayor\'ia de soluciones descritas en la literatura consisten en herramientas creadas ad-hoc para ofrecer una determinada funcionalidad con mayor o menor grado de interactividad en la composici\'on del resultado de la agregaci\'on por parte del usuario final. Estas soluciones trabajan con un conjunto pre-determinado de servicios web \textit{REST} que no pueden ser modificados.\\
Algunas otras herramientas han intentado aportar un mayor grado de genericidad en la construcci\'on de \textit{mashups} explotando las caracter\'isticas de los servicios \textit{REST}, como la homogeneidad en el acceso a la informaci\'on. El trabajo m\'as destacado es el del \textit{Yahoo} con la construcci\'on de \textit{Yahoo Pipes} \cite{yahoo_pipes} una herramienta para la construcci\'on de \textit{mashups} web de una forma sencilla por parte del usuario final. La caracter\'istica m\'as interesante de Yahoo Pipes, reside en el uso de un meta servicio de datos capaz de adaptar cualquier servicio web que se ajuste a la interfaz uniforme \textit{HTTP} y que use \textit{JSON} como el formato de datos para la representaci\'on de los recursos expuestos. La interpretaci\'on de la sem\'antica de los datos \textit{JSON} obtenidos desde el servicio debe ser aportada por el usuario final construyendo la \textit{mashup} usando alguno de los elementos predefinidos que el sistema provee.\\
Por \'ultimo, otra aportaci\'on interesante en el desarrollo de \textit{mashups} web es el de la construcci\'on de lenguajes de programaci\'on o \textit{scripting} \cite{sabbouh2007web} espec\'ificamente dise\~nados para construir \textit{mashups} web a trav\'es de la integraci\'on de diferentes servicios \textit{REST}.\\
En dichos lenguajes, las operaciones primitivas que se pueden aplicar a un recurso \textit{REST} a trav\'es de la interfaz \textit{HTTP}, as\'i como la petici\'on de determinadas representaciones y las operaciones necesarias para agregar estas representaciones son transformadas en primitivas computacionales de alto nivel en el lenguaje de programaci\'on, de forma tal, que la composici\'on de servicios puede ser descrita de una forma sencilla a trav\'es del lenguaje de programaci\'on. Este programa puede ser a continuaci\'on ejecutado de forma autom\'aticamente por un agente software dotado de un interprete para dicho lenguaje de programaci\'on.\\

\subsection{\textit{HATEOAS}, \textit{Hypermedia} como el Motor del Estado de la Aplicaci\'on}

Un concepto clave que marca la diferencia entre el conjunto de especificaciones t\'ecnicas \textit{WS-*} y soluciones para la construcci\'on de servicios web basados en la invocaci\'on remota de procedimientos (\textit{RPC}) frente a los principios arquitecturales \textit{REST} es el del uso de los tipos de datos y los enlaces para mantener el estado de la aplicaci\'on.\\
Este concepto aparece ya en la tesis original sobre \textit{REST} de Roy Fielding \cite{fielding2000representational}, pero ha venido a consolidarse m\'as tarde bajo el acr\'onimo \textit{HATEOAS} (\textit{hypermedia} como el Motor del Estado de la Aplicaci\'on). La consecuencia b\'asica de este principio para la construcci\'on de \textit{APIs} de servicios web \textit{REST} es que la mayor\'ia de elementos del servicio, por ejemplo el protocolo de comunicaci\'on y su sem\'antica, debe ser fijos y est\'andar, en este caso el protocolo \textit{HTTP} y la sem\'antica predefinida para las operaciones del protocolo. El \'unico elemento sobre el que puede actuar el dise\~nador de la \textit{API} es sobre el tipo de datos asignado a las diferentes representaciones de los recursos expuestos.\\
Desde este principio de arquitectura, el dise\~no de \textit{APIs} es equivalente al dise\~no de tipos de datos, y conociendo como manipular un determinado tipo de datos, cualquier agente web deber\'ia ser capaz de utilizar una \textit{API} \textit{REST} partiendo de una \textit{URI} inicial, dado que el \textit{REST}o de elementos de la interfaz van a ser homog\'eneos y constantes respecto a cualquier otra \textit{API} \textit{REST}. Esto \'ultimo se logra a trav\'es de una segunda consecuencia b\'asica derivada del principio de dise\~no \textit{HATEOAS}, cualquier cambio en el estado de la aplicaci\'on se produce en el lado del cliente y es el resultado de seguir un enlace en la representaci\'on del recurso obtenida. De este modo, el servicio ofrece las opciones necesarias al cliente para que elija el siguiente estado de la aplicaci\'on de una forma est\'andar y el cliente debe limitarse a extraer estos enlaces, que tambi\'en pueden incluir otros elementos de control, como en un formulario \textit{HTML} y seleccionar el siguiente estado.\\
Una l\'inea de investigaci\'on fruct\'ifera dentro de la comunidad acad\'emica \textit{REST} ha consistido en la aplicaci\'on de las constricciones que el principio arquitect\'onico \textit{HATEOAS} impone a diferentes problemas en el desarrollo de servicios web. En la secci\'on anterior hemos visto como de forma impl\'icita \textit{HATEOAS} se ha intentado aplicar a la orquestaci\'on y coreograf\'ia de servicios web \textit{REST} a trav\'es del desarrollo de lenguajes espec\'ificos que ten\'ian en cuenta la presencia enlaces en las representaciones de los recursos para descubrir mecanismos para manipular dichos recursos o recursos asociados \cite{alarcon2011hypermedia}.\\
Otro ejemplo de la importancia de \textit{HATEOAS} para la definici\'on de servicios web \textit{REST} es evidente en la sucesi\'on de art\'iculos sobre como aplicar el principio \textit{HATEOAS} a \textit{APIs} ya existentes para transformarlas en \textit{APIs} consistentes con los principios arquitect\'onicos \textit{REST} tal y como se muestra en \cite{liskin2011teaching}.

\section{Web Sem\'antica}

La Web Sem\'antica ha recorrido un largo trayecto desde su concepci\'on hace m\'as de diez a\~nos por parte de Tim Berners Lee como una versi\'on de la web centrada en datos f\'acilmente procesables de forma automatizada por agentes software, en vez de una web \'unicamente de documentos para ser interpretados por usuarios humanos.\\
La visi\'on \'ultima de la Web Sem\'antica, en la que la inferencia l\'ogica jugar\'ia un papel destacado en la forma en que agentes software llevar\'ian a cabo tareas determinadas usando los datos disponibles en la web, no ha sido alcanzada en su totalidad. Sin embargo, en estos a\~nos de investigaci\'on, numerosos resultados, est\'andares y tecnolog\'ias relevantes para la investigaci\'on expuesta en esta tesis han sido desarrollados en el seno de la comunidad de investigaci\'on sobre Web Sem\'antica. 
En los siguientes apartados, se har\'a un repaso de los resultados m\'as importantes.

\subsection{Est\'andares Sem\'anticos}

La iniciativa Web Sem\'antica ha sido liderada y llevada a cabo en su mayor parte al amparo del consorcio Web (\textit{W3C}). Esto supone que la investigaci\'on m\'as importante sobre Web Sem\'antico ha cristalizado en el desarrollo de est\'andares web de la \textit{W3C}. Algunos de estos est\'andares han tenido una gran influencia, no s\'olo dentro de la comunidad Web Sem\'antica sino que ha tenido impacto en otras \'areas de investigaci\'on relacionadas y en la industria en algunos casos de uso en los que era necesario trabajar con datos estructurados.\\
Un primer est\'andar es \textit{RDF} (\textit{Resource Description Framework} o \textit{Framework} para la Descripci\'on de Recursos). \textit{RDF} describe simult\'aneamente un modelo de datos de una gran genericidad, consistente en un multi grafo dirigido, una serializaci\'on de dicho modelo de datos como una serie de expresiones sujeto-predicado-objeto y una sem\'antica precisa para dicho modelo de datos \cite{hayes2004rdf}.\\
\textit{RDF} se encuentra en el nivel m\'as b\'asico de la pila de est\'andares propuestos por la comunidad Web Sem\'antica, ya que es utilizado como el modelo de datos sobre el que se construyen el \textit{REST}o de est\'andares sem\'anticos.\\
Un grafo \textit{RDF} puede ser expresado usando diferentes sintaxis estandarizadas por la \textit{W3C}. La primera sintaxis propuesta estaba basada en \textit{XML} (sintaxis \textit{RDF/XML}) \cite{beckett2004rdf} y presenta una notable complejidad para los usuarios de dicha sintaxis, ya sea a la hora de generar la serializaci\'on de un grafo \textit{RDF} o a la hora de construir un \textit{parser} capaz de reconstruir un grafo \textit{RDF} a partir de un documento \textit{RDF/XML}. Un \'area de investigaci\'on destacable en estos a\~nos ha consistido en la b\'usqueda de mejores sintaxis para la serializaci\'on de grafos \textit{RDF}, teniendo como resultado algunas propuestas de sintaxis simples para \textit{RDF} como \textit{N3} \cite{n3} o \textit{Turtle} \cite{turtle}.\\
Otro est\'andar sem\'antico de gran importancia es \textit{SPARQL} (\textit{SPARQL Protocol and RDF Query Language}) \cite{sparql}, un lenguaje que permite realizar consultas sobre un grafo \textit{RDF} mediante expresiones consistentes en patrones de expresiones sujeto-predicado-objeto donde variables pueden ser intercaladas. \textit{SPARQL} es un lenguaje de consulta muy expresivo, que permite extraer partes de un grafo \textit{RDF} o informaci\'on particular de dicho grafo con una sintaxis similar a la de lenguajes de consulta para bases de datos relacionales como \textit{SQL}.\\
\textit{SPARQL} ha sufrido una evoluci\'on importante desde su especificaci\'on inicial. Se han elaborado resultados acerca de la sem\'antica formal y la complejidad computacional de las consultas \textit{SPARQL} \cite{perez2006semantics} y el est\'andar se ha extendido en su versi\'on \textit{SPARQL 1.1 UPDATE} \cite{sparql11} para dar soporte no solo a la recuperaci\'on de informaci\'on del grafo \textit{SPARQL}, sino a la modificaci\'on del grafo mediante consultas espec\'ificas para la actualizaci\'on, la inserci\'on y el borrado de tripletes en un grafo \textit{RDF}.\\
Otro conjunto clave de especificaciones sem\'anticas lo constituyen los lenguajes para definir ontolog\'ias sobre grafos \textit{RDF}. \textit{OWL} (\textit{Web Ontology Language}) \cite{owl} es el lenguaje est\'andar propuesto por la \textit{W3C} para la construcci\'on de vocabularios y ontolog\'ias con los que describir recursos web. \textit{OWL} extiende la sem\'antica propia de \textit{RDF} y de \textit{RDFS} (\textit{RDF schema}) \cite{rdfs} con primitivas ontol\'ogicas provenientes de la teor\'ia formal de la L\'ogica Descriptiva \cite{owl_dl_reduction}.\\
En la primera versi\'on del lenguaje, se especifican diferentes perfiles, que suponen un balance inverso entre expresividad, n\'umero de primitivas que se pueden usar en dicho perfil y complejidad algor\'itmica a la hora de llevar a cabo tareas de inferencia (clasificaci\'on, comprobaci\'on de consistencia, etc) sobre una ontolog\'ia expresada usando ese perfil. En la segunda versi\'on del lenguaje, \textit{OWL 2} \cite{owl2}, los perfiles se han reorganizando, asignando las primitivas en base a casos de uso particulares (\textit{QL}, \textit{EL}, \textit{RL}) \cite{owl2_profiles} y no tanto un incremento simple de menor a mayor complejidad y expresividad.\\
El lenguaje de consulta \textit{SPARQL} tambi\'en puede ser extendido \cite{sirin2007sparql}, \cite{glimm2009sparql}, para tener en cuenta el uso de un determinado perfil OWL, de tal manera que el conjunto de tripletes en el grafo \textit{RDF} que van a ser recuperados por una consulta \textit{SPARQL} se vea aumentado mediante el uso de la inferencia l\'ogica disponible para el perfil de \textit{OWL} seleccionado.
La implementaci\'on de soluciones tecnol\'ogicas que permitan implementar las especificaciones tanto de \textit{RDF}, como de \textit{SPARQL} y \textit{OWL} ha sido otro importante l\'inea de investigaci\'on y desarrollo durante los \'ultimos a\~nos de evoluci\'on de la iniciativa Web Sem\'antica.\\
Se han propuesto diferentes soluciones tecnol\'ogicas para cada uno de los anteriores est\'andar: repositorios de tripletes \textit{RDF}, motores de consulta \textit{SPARQL} y razonadores l\'ogicos OWL\\. 
De entre ellas destacamos algunas interesantes arquitecturas para el almacenamiento de grafos \textit{RDF} as\'i como de motores de consulta \textit{SPARQL} \cite{yars} \cite{rohloff2010high}.\\

\subsection{Servicios Web Sem\'anticos}

La anotaci\'on sem\'antica de los meta-datos asociados a un servicio web, de tal forma que sea posible que un agente software consuma de forma automatizada dicho servicio es uno de los problemas que han intentando solucionar las diferentes especificaciones para la construcci\'on de servicios web. La soluci\'on que ofrece las tecnolog\'ias web sem\'antica consiste en ontolog\'ias y modelos de servicio est\'andar que se pueden combinar con los est\'andares \textit{WS-*} o arquitecturas \textit{REST} para solucionar el problema de la inter-operabilidad entre servicios.\\
Algunas de as principales propuestas de ontolog\'ias para la descripci\'on de servicios web sem\'anticos son las siguientes \cite{lanthaler2010semantic}:

\begin{itemize}

\item \textbf{\textit{OWL-S} \cite{martin2004owl}}: Es propuesta de ontolog\'ia construida usando el est\'andar sem\'antico de la \textit{W3C} \textit{OWL}, para la anotaci\'on de servicios web. \textit{OWL-S} incluye vocabularios para la descripci\'on del perfil del servicio (\textit{service profile}), de tal forma que el servicio pueda ser descubierto por agentes software, un vocabulario para describir el modelo de proceso del servicio (\textit{service process model}), los detalles operacionales concretos del servicio, incluyendo aspectos como el n\'umero de operaciones y los tipos de entrada y salida de los argumentos.  Por \'ultimo \textit{OWL-S} incluye un vocabulario para la descripci\'on del \textit{grounding} del servicio, es decir, como los elementos del proceso de modelo se implementan como elementos del est\'andar de servicios web \textit{WS-*}, por ejemplo una descripci\'on \textit{WSDL}. La principal cr\'itica recibida por \textit{OWL-S} proviene de lo est\'atico de su propuesta, ya que no recoge los elementos necesarios para describir variaciones temporales en las descripciones de los servicios ni un n\'umero arbitrario de operaciones no relacionadas.

\item \textbf{\textit{WSMO} \cite{wsmo}}: Es un intento de construir una especificaci\'on formal completo que pueda ser usada en la descripci\'on de servicios web sem\'anticos. \textit{WSMO} incluye un un modelo formal y lenguaje de descripci\'on (\textit{WSML}) y una implementaci\'on de referencia para un entorno de ejecuci\'on de los servicios \textit{WSMO} (\textit{WSMX}). \textit{WSMO} permite definir diferentes aspectos del entorno de uso de servicios web, incluyendo el dominio de aplicaci\'on de los servicios, los objetivos que los agentes software consumiendo los servicios necesitan completar y las descripciones de los servicios que pueden satisfacer esos objetivos. Una de las principales cr\'iticas recibidas por \textit{WSMO} es que su especificaci\'on se llevo a cabo sin tener en cuenta los est\'andares sem\'anticos como \textit{OWL}, lo que supuso un inconveniente para que \textit{WSMO} se convirtiese en un est\'andar web \textit{W3C}. Una versi\'on reducida del est\'andar \textit{WSMO-Lite} \cite{wsmo_lite} fue creada para solventar estos inconvenientes y ofrecer mejor compatibilidad con los est\'andares sem\'anticos \textit{W3C}.

\item \textbf{\textit{SAWSDL} \cite{sawsdl}}: Es un intento por parte de la \textit{W3C} de especificar una ontolog\'ia est\'andar para la anotaci\'on de servicios web definidos usando \textit{WSDL} con informaci\'on sem\'antica. Extensiones al est\'andar describen elementos claves presentes en otros intentos previos, como el modelo de referencia para los servicios y vocabularios para el lifting y lowering de los servicios, es decir la relaci\'on existente entre los elementos descritos mediante la ontolog\'ia de \textit{SAWSDL} y los tipos b\'asicos \textit{XML} que se intercambiar\'an entre cliente y servicio cuando el servicio es consumido.

\end{itemize}

As\'i mismo, la comunidad Web Sem\'antica tambi\'en ha intentado incluir elementos sem\'anticos en los meta-datos asociados a servicios web \textit{REST}.\\
Algunas de las principales propuestas aparecidas en los \'ultimos a\~nos son:

\begin{itemize}

\item \textbf{\textit{hRESTS} \cite{hrests}}: Es una propuesta para transformar la documentaci\'on \textit{HTML} ad-hoc para humanos asociadas con las \textit{APIs} \textit{REST} en meta-datos procesables autom\'aticamente por agentes software usando microformatos. Los microformatos \cite{microformats} son un intento de incluir meta-datos en documentos \textit{XHTML} usando determinados atributos y etiquetas \textit{HTML} siguiendo una serie de convenciones. Diferentes vocabularios de mciroformatos (para informaci\'on personal, tareas, \textit{curriculum vitae}, etc) definen como los atributos deben ser usados y que valores dichos atributos deben tener. \textit{hRESTS} consiste en una de dichas especificaciones que permite marcar los elementos b\'asicos del modelo \textit{REST} de servicios en la documentaci\'on, para que un agente software capaz de procesar la especificaci\'on \textit{hRESTS}, pueda extraerlas y consumirlas.

\item \textbf{\textit{MicroWSMO} \cite{microwsmo}}: Es una especificaci\'on que intenta anotar servicios web \textit{REST} usando un modelo de servicio basado en \textit{SAWSDL} \cite{sawsdl} (modelo de servicio, \textit{lifting} y \textit{lowering}) usando \textit{hRESTS} para anotar la documentaci\'on \textit{HTML} asociada a los servicios. La gran ventaja de usar el vocabulario propuesto por \textit{MicroWSMO} es que al ser una ontolog\'ia compatible con \textit{SAWSDL}, \textit{MicroWSMO} puede usarse como un mecanismo para integrar servicios web \textit{REST} con servicios \textit{WS-*}.

\item \textbf{\textit{SA-REST}} \cite{sarest}: Una propuesta similar a \textit{MicroWSMO} para anotar la documentaci\'on \textit{HTML} asociada a un servicio web \textit{REST}. La principal diferencia entre \textit{SA-REST} y \textit{MicroWSMO} reside en el hecho de que \textit{SA-REST} utiliza \textit{RDFa} \cite{rdfa}, un est\'andar \textit{W3C} que ser\'a revisado en la pr\'oxima secci\'on para a\~nadir meta-datos a un documento \textit{HTML} en vez de usar un mecanismo no est\'andar como son los microformatos. En cuanto a las naturaleza de las anotaciones tambi\'en son muy similares a las propuestas por \textit{MicroWSMO}, consistentes en una descripci\'on del modelo de servicio basada en componentes como tipos de entrada, salida, operaciones, acciones \textit{HTTP}, errores y los mecanismos de \textit{lifting} y \textit{lowering}.

\end{itemize}

\subsection{\textit{ICV} Validaciones de Restricciones de Integridad}

Un \'ultimo desarrollo aparecido dentro de la comunidad Web Sem\'antica es el de un mecanismo para interpretar una ontolog\'ia construida usando est\'andares sem\'anticos como \textit{OWL} y \textit{RDF Schema} como restricciones de integridad de datos que pueden ser validadas (\textit{ICV}) \cite{tao2010integrity}, \cite{motik2009bridging}.\\
La base del mecanismo supone cambiar los axiomas fundamentales utilizados para realizar inferencia l\'ogica sobre una ontolog\'ia: el Asunci\'on de Mundo Abierto (\textit{OWA}) y la ausencia de Asunci\'on de Nombres \'unicos (\textit{UNA}) \cite{sirin2008opening}. Ambos axiomas, heredados de la L\'ogica Descriptiva, permiten a un razonador \textit{OWL} a\~nadir nuevas inferencias cuando realiza la interpretaci\'on de un modelo dado por una ontolog\'ia \textit{OWL}.\\
La Asunci\'on de Mundo Abierto implica que una proposici\'on no puede ser inferida como falsa si no se cuenta con la informaci\'on para probarlo de esta manera, por el contrario, el razonador puede a\~nadir las proposiciones necesarias para que la interpretaci\'on sea verdadera. Por su parte la ausencia de Asunci\'on de Nombres \'unicos significa que dos identificadores diferentes en dos proposiciones del modelo pueden interpretarse como referentes al mismo individuo para hacer la interpretaci\'on consistente. Estas asunciones son muy \'utiles en el contexto de un lenguaje de ontolog\'ias dise\~nado para ser usado en la web, un entorno altamente distribuido donde la informaci\'on puede ser agregada de forma gradual desde diferentes servicios, pero contrastan marcadamente con las bases te\'oricas empleadas en otros sistemas de informaci\'on m\'as habituales, como los sistemas gestores de datos basados relacionales o en sistemas l\'ogicos derivados de Datalog \cite{motik2006can}. En dichos sistemas, la Asunci\'on de Mundo Cerrrado (\textit{CWA}) es usada en vez de la Asunci\'on de Mundo abierto y al mismo tiempo se observa la Asunci\'on de Nombres \'unicos.\\
Esta \'ultima interpretaci\'on de un modelo de datos es la que resulta conveniente para validar la entrada de datos en un sistema en el que dichos datos deben cumplir unos requisitos m\'inimos de integridad. El desarrollo de \textit{ICV} para \textit{OWL} ofrece dicha interpretaci\'on para una ontolog\'ia que puede ser usada al mismo tiempo para validar la integridad de los datos si se usa la interpretaci\'on \textit{ICV} basada en CWA y UNA o para inferir nueva informaci\'on si se usa la interpretaci\'on cl\'asica de \textit{OWL} basada en OWA y ausencia de UNA.
Esta propuesta de a\~nadir \textit{ICV} a \textit{OWL} ha sido recientemente implementada satisfactoriamente en algunos sistemas sem\'anticos comerciales como \textit{Pellet-ICV}, \textit{Stardog} y se pretende que sea aceptada como una propuesta de est\'andar \textit{W3C}.

\section{Datos Enlazados Abiertos (\textit{Open Linked Data})}

A pesar de la ingente cantidad de trabajo realizada por la comunidad Web Sem\'antica, especialmente en las \'area de la inferencia l\'ogica, la pila de est\'andares sem\'anticos producidos por la \textit{W3C} no han encontrado un grado significativo de adopci\'on en la industria del desarrollo web \cite{pedantic}. De un modo especialmente significativo, las propuestas de anotaci\'on de servicios web \textit{REST} han quedado circunscritas al \'ambito acad\'emico sin aplicaci\'on real.\\
Como respuesta a esta falta de adopci\'on surge la iniciativa \textit{OLD}, Datos Enlazados Abiertos (\textit{Open Linked Data}) \cite{old_story}. La comunidad \textit{OLD} intenta reorientar el enfoque de la investigaci\'on en Web Sem\'antica de las capas de razonamiento l\'ogico hacia las capas m\'as b\'asica centradas en la disponibilidad de meta-datos procesables autom\'aticamente en la web, mezclando aspectos b\'asicos de las arquitecturas \textit{REST} con los conceptos propios de la Web Sem\'antica \cite{page2011rest}:

\begin{itemize}

\item La importancia dada a los \textit{URIs}, no s\'olo como un identificador en una ontolog\'ia, sino como un mecanismo que puede ser desreferenciado mediante una petici\'on \textit{HTTP} para acceder a la informaci\'on de un recurso identificado por ese \textit{URI} \cite{sauermann2011cool}.

\item La necesidad de incluir enlaces entre recursos mediante el uso de \textit{URIs} \cite{hausenblas2009exploiting}, aspecto no siempre expl\'icito en los est\'andares sem\'anticos, donde los \textit{URIs} se contemplaban m\'as como identificadores opacos que como mecanismos de desreferencici\'on.

\item Uso de las posibilidades propias del protocolo \textit{HTTP}, desde una perspectiva arquitect\'onica \textit{REST}, para, por ejemplo, decidir diferentes serializaciones de un grafo \textit{RDF} asociado a un \textit{URI} usando negociaci\'on de conenido \cite{bizer2007publish}.

\item Un enfoque pragm\'atico centrado en ofrecer soluciones v\'alidas para la industria de desarrollo web y que puedan ser adoptadas de forma incremental en una arquitectura que siga los principios de dise\~no \textit{REST} \cite{heath2011linked}.

\end{itemize}

El resultado de esta mezcla entre conceptos propios de la Web Sem\'antica y principios arquitect\'onicos \textit{REST}, es una vuelta a la concepci\'on de la Web Sem\'antica como una versi\'on de la arquitectura actual de la web, basada en el protcolo HTTP y los conceptos arquitect\'onicos REST, pero centrada en el intercambio de datos enlazados entre servicios web.\\
En los siguientes apartados, enumeraremos algunos de los principales problemas que la comunidad acad\'emica relacionada con la iniciativa de Datos Enlazados Abiertos ha abordado en los \'ultimos a\~nos.

\subsection{Marcado Sem\'antico}

Uno de los problemas a los que se ha intentado encontrar soluci\'on desde la comunidad de Datos Enlazados Abiertos consiste en la b\'usqueda de mecanismos con los que insertar el contenido sem\'antico estructurado en los documentos \textit{HTML} generados para ser visualizados por humanos.\\
La meta \'ultima que se persigue con estos mecanismos es la de ofrecer dos versiones de la informaci\'on en el mismo documento \textit{HTML}, el contenido destinado a las personas, constituido por el conjunto de texto en lenguaje natural, im\'agenes y otros elementos de hipermedia, y al mismo tiempo, ofrecer una descripci\'on de la sem\'antica de esos contenidos que pueda ser autom\'aticamente procesada por agentes software.\\
La primera soluci\'on propuesta a este problema, no proviene directamente de la comunidad de Datos Abiertos Enlazados, sino que es una propuesta surgida en el \'ambito del desarrollo web. Esta propuesta viene dad por la comunidad de desarrollo de Microformatos \cite{microformats}, anteriormente mencionada.\\
Los Microformatos consisten en esquemas sencillos que combinan etiquetas \textit{HTML}, atributos de esas etiquetas y ciertos valores para dichos atributos, con los que se intenta insertar contenido sem\'antico en el c\'odigo \textit{HTML} de una p\'agina web. Existen diversas propuestas de microformatos para diferentes tipos de contenido sem\'antico, incluyendo datos sobre personas, organizaciones, fechas, eventos, licencias etc.\\
El uso de Microformatos ha cosechado un \'exito relativo dentro de la comunidad de desarrollo web, sin embargo, han sido, casi desde el inicio de su desarrollo, objeto de cr\'itica por parte de la comunidad Web Sem\'antica. Los principales inconvenientes que se han esgrimido en contra del uso de Microformatos \cite{graf2007rdfa} como mecanismo con el que a\~nadir contenido sem\'antico a documentos \textit{HTML} incluyen su incompatibilidad con la pila de est\'andares sem\'anticos propuesta por la \textit{W3C} as\'i como el hecho de que la no observancia de ciertos est\'andares, como el uso de \textit{URIs} y la preferencia por atributos en texto plano, provoca graves problemas de extensibilidad, que impiden que nuevos Microformatos puedan usarse de forma arbitraria para a\~nadir contenido sem\'antico a nuevos dominios de aplicaci\'on.\\
Estas cr\'iticas fueron recogidas por una parte de la comunidad Web Sem\'antica primero, y de la comunidad de Datos Enlazados Abierto despu\'es, para proponer un mecanismo alternativo de marcado sem\'antico para documentos \textit{HTML}, extensible y que se integrase de forma satisfactoria con el \textit{REST}o de tecnolog\'ias sem\'anticas de la \textit{W3C}. Este nuevo est\'andar conocido como \textit{RDFa} (\textit{Resource Description Framework in Attributes}) \cite{rdfa}, permite la inserci\'on de un grafo \textit{RDF} dentro de un documento \textit{HTML}. Para ello, \textit{RDFa} usa una t\'ecnica similar a la propuesta por los Microformatos, usando algunos de los atributos \textit{HTML} para se\~nalar sujetos, predicados y objetos dentro de los tripletes \textit{RDF} que conforman el grafo que se est\'a insertando en el documento \textit{HTML}. La principal diferencia respecto al uso de Microformatos, es que en lugar de usar texto plano para describir los atributos del vocabulario anotado, \textit{RDFa} hace uso de \textit{URIs} y de una notaci\'on especifica para \textit{URIs} abreviados que pueden insertarse como valores de atributos \textit{HTML} denominada \textit{CURIEs} \cite{curies}.\\
El resultado es que cualquier grafo \textit{RDF} use el tipo de vocabulario que use, puede ser insertado en un document \textit{XHTML} usando \textit{RDFa}, con lo que se consigue una gran flexibilidad en el marcado de documentos.\\
A pesar de su expresividad como mecanismo de anotado sem\'antico, \textit{RDFa} ha sufrido cr\'iticas debido a ser considerado como de una complejidad demasiado elevada para encontrar una adopci\'on significativa por parte de la comunidad de desarrolladores web. En respuesta a esta cr\'itica algunas de las principales empresas web como \textit{Google}, \textit{Yahoo} y \textit{Microsoft} han propuesto, dentro del proceso de estandarizaci\'on de \textit{HTML5},  \textit{Microdata} \cite{microdata}, un mecanismo alternativo a \textit{RDFa} que intenta ofrecer una alternativa m\'as simple para la inserci\'on de contenido sem\'antico en documentos \textit{HTML}. Microdata es una propuesta de est\'andar que a\~nade una serie de atributos nuevos a \textit{HTML} con los que describir contenido sem\'antico, de una forma extensible, cumpliendo algunos est\'andares \textit{W3C}, como el uso de \textit{URIs}, pero sin ofrecer toda la capacidad expresiva de \textit{RDFa}. Microdata usa un modelo de datos, consistente en un \'arbol de relaciones entre elementos del documento \textit{HTML}, diferente al modelo de grafo \textit{RDF} que propone \textit{RDFa}. A pesar de estas diferencias, algunos resultados recientes \cite{hickson2012microdata} han propuesto mecanismos para transformar un \'arbol de relaciones Microdata anotado en un documento \textit{HTML} en un grafo \textit{RDF}, con lo que ser\'ia posible integrar documentos Microdata con el \textit{REST}o de est\'andares sem\'anticos \textit{W3C}. Del mismo modo que exiten un determinado n\'umero de vocabularios propuestos para ser usados con Microformatos, algunos vocabularios han sido propuestos por las compa\~n\'ias detr\'as del esfuerzo de estandarizaci\'on de Microdata para determinados dominios de aplicaci\'on \cite{ronallo2012html5}. Dichos vocabularios han sido tambi\'en transformados en ontolog\'ias \textit{RDF} que pueden ser utilizadas en documentos anotados con \textit{RDFa}, consiguiendose de esta manera un cierto grado de inter-operabilidad entre ambos est\'andares.\\

\subsection{Desreferenciaci\'on de recursos web}

Como se ha mencionado anteriormente, una de las principales aportaciones que la iniciativa Datos Enlazados Abiertos a aportado al enfoque tradicional de investigaci\'on en Web Sem\'antica es el da la importancia de los \textit{URIs} como identificadores de recursos que pueden ser obtenidos mediante operaciones \textit{HTTP}, equipar\'andolos de esta manera a recursos expuestos por servicios web, tal y como son descritos en las arquitecturas \textit{REST}. El problema surge a la hora de encontrar una representaci\'on adecuada que devolver a un agente web que intenta desreferenciar una \textit{URI} asociada a un recurso presente en una ontolog\'ia \textit{RDF} y que puede representar un concepto abstracto y no desreferenciable.\\
Este problema es conocido dentro de la comunidad de Datos Enlazados Abiertos como \textit{HTTP Range-14} \cite{fielding2005httprange}. Diversas soluciones han sido propuestas para intentar abordar este problema.\\
En primer lugar, se considera que recursos de informaci\'on, como documentos \textit{HTML} deben ser identificados por \textit{URIs} sin acabar en un fragmento introducido por el car\'acter ``\#'', mientras que \textit{URIs} identificando recursos abstractos no de informaci\'on y no desreferenciable, deber\'ian usar \textit{URIs} con fragmentos. Esto permitir\'ia al servicio ofreciendo el recurso tomar una acci\'on adecuada y responder con un c\'odigo de estado \textit{HTTP} y cabeceras adecuadas para que el agente web pueda acceder a una representaci\'on de dicho recurso.\\
En concreto, si el recurso es un recurso de informaci\'on y puede ser desreferenciado, el servicio debe devolver un c\'odigo \textit{200} pero si el recurso es no de informaci\'on y no puede ser desreferenciado, el servicio debe devolver un c\'odigo de redirecci\'on \textit{300} y ofrecer una cabecera con informaci\'on sobre recursos de informaci\'on asociados, que supongan una representaci\'on adecuada para el concepto abstracto que se intenta desreferenciar.  Esta soluci\'on cuenta con el inconveniente de la gran cantidad de ontolog\'ias que ya han sido propuestas y que son ampliamente empleadas, que no utilizan el formato correcto para las \textit{URIs} de conceptos abstractos.


\subsection{RESTful \textit{SPARQL}}

Uno de los problemas a los que se ha enfrentado la comunidad de Datos Abiertos Enlazados  ha consistido en exponer como servicios web ingentes cantidades de informaci\'on codificada como grafos \textit{RDF} procesables autom\'aticamente. Como ya se ha mencionado en este documento, \textit{SPARQL} constituye el est\'andar sem\'antico b\'asico usado para recuperar informaci\'on en un grafo \textit{RDF}. Una de las primeras aproximaciones utilizadas por la comunidad de Datos Abiertos Enlazados consisti\'o en exponer directamente grafos \textit{RDF} a trav\'es de puntos \textit{SPARQL} accesibles a trav\'es del protocolo \textit{HTTP}. Una propuesta de est\'andar web fue creada al respecto, especificando un protocolo \cite{sparql_protocol} que permit\'ia intercambiar consultas \textit{SPARQL} y resultados entre un punto \textit{SPARQL} y un cliente \textit{HTTP}.\\
Sin embargo, este mecanismo de intercambio de consultas \textit{SPARQL} entra en conflicto con los principios arquitect\'onicos b\'asicos \textit{REST}, ya que ignora el concepto de recurso latente en est\'andares sem\'anticos como \textit{RDF} y propone por el contrario un mecanismo basado en la invocaci\'on remota de procedimientos que se traducen en consultas \textit{SPARQL} y la correspondiente respuesta \textit{XML}.\\
Una posible soluci\'on para reconciliar el uso de \textit{SPARQL} como un mecanismo viable para exponer grafos \textit{RDF} a trav\'es de una interfaz \textit{HTTP} viene de la mano del concepto de grafo con nombre introducida por la especificaci\'on de \textit{SPARQL} \cite{wilde2009restful}. \textit{SPARQL} permite asociar un \textit{URI} identificativo a un determinado grafo \textit{RDF} y ejecutar una consulta sobre uno o m\'as grafo. De esta manera, la unidad b\'asica de las arquitecturas \textit{REST}, el recurso web asociado a un \textit{URI}, se asimilar\'ia a la de grafo \textit{RDF} con nombre en \textit{SPARQL}, sobre el que se ejecutar\'ia diferentes operaciones \textit{HTTP}, cuya sem\'antica quedar\'ia recogida en las correspondientes consultas \textit{SPARQL} y \textit{SPARQL Update}, para recuperar informaci\'on del grafo, eliminarle o modificarla.
Esta l\'inea de trabajo se ha formalizado en una propuesta de est\'andar web en la \textit{W3C}, conocida como Protocolo de Almacenamiento de Grafos para \textit{SPARQL} 1.1 \cite{ogbuji2011sparql} o la arquitectura de servicios Pubby \cite{cyganiak2008pubby}.\\
Dicha propuesta de est\'andar propone una traducci\'on de las operaciones \textit{HTTP} a diferentes consultas \textit{SPARQL}  y \textit{SPARQL} Update para que sean ejecutadas sobre grafos \textit{RDF} con nombre con una \textit{URI} asociada y que se expondr\'ia a trav\'es del protocolo \textit{HTTP} como recursos \textit{REST}.\\
Una \'ultima alternativa m\'as simple para la construcci\'on de una interfaz \textit{HTTP} para un grafo \textit{RDF} consiste en la traducci\'on del grafo entidades atributo-valor. En este tipo de \textit{APIs}, el servicio oculta la naturaleza \textit{RDF} de los datos expuestos, as\'i como el modelo de datos \textit{RDF}, traduciendo conjuntos de tripletes \textit{RDF} a objetos que contienen pares clave valor. Estos objetos son codificados en los par\'ametros de las peticiones \textit{HTTP} enviadas por los clientes y son devueltos en las respuestas \textit{HTTP}, normalmente serializados como objetos \textit{JSON} con atributos planos. Ejemplos de estas \textit{APIs} son la Propuesta de \textit{API} para Datos Enlazados (\textit{LD-API}) \cite{ld_api} y las versiones que usan \textit{RDF} como formato \'ultimo de datos de la \textit{API} para un Protocolo de Datos Abiertos \cite{fatland2011open}. Las interfaces basadas en entidades atributo-valor son compatibles con los principios arquitect\'onicos \textit{REST} y tienen una interfaz muy familiar para cualquier desarollador web. Normalmente introducen mecanismos ad-hoc para lidiar con problemas pr\'acticos propios del desarrollo web, como la paginaci\'on de recursos en colecciones. Por otro lado, incurren en ciertos problemas desde el punto de vista de las recomendaciones sobre Datos Enlazados Abiertos, como el uso de \textit{URIs} ocultos en los atributos de los objetos, lo que puede hacer imposible el enlazado efectivo entre diferentes \textit{APIs}.\\

\subsection{\textit{JSON} Enlazado}

Un \'ultimo desarrollo interesante encontrado dentro de la comunidad de Datos Enlazados Abiertos, consiste, en la b\'usqueda de serializaciones v\'alidas para grafos \textit{RDF} como objetos \textit{JSON} \cite{json}.\\
Como ya se ha mencionado, \textit{JSON} es el formato de intercambio preferido por los desarrolladores web en la actualidad a la hora de desarrollar \textit{APIs} de datos. Sin embargo, \textit{JSON} cuenta con graves inconvenientes como serializaci\'on de recursos, entre ellos, la imposibilidad de identificar al recurso que ha generado el objeto \textit{JSON} dentro del objeto de una forma est\'andar, as\'i como la imposibilidad de denotar de una forma est\'andar \textit{URIs} en los pares claves valor que conforman el documento \textit{JSON}.\\
\textit{JSON-LD} (\textit{JSON} para Datos Enlazados) \cite{jsonld} ha sido una de las diferentes propuestas \cite{alexander2008rdf} para insertar grafos \textit{RDF} dentro de documentos \textit{JSON}. \textit{JSON-LD} propone atributos especiales que denotan la identidad del recurso siendo serializado en el objeto, as\'i como atributos que permiten identificar qu\'e claves y qu\'e valores del objeto son \textit{URIs} y cu\'ales alg\'un otro tipo de dato.\\
\textit{JSON-LD} tiene un dise\~no muy flexible que permite serializar cualquier tipo de grafo \textit{RDF} en una colecci\'on de objetos \textit{JSON}, pero al mismo tiempo, el uso de valores convencionales por defecto, permite que en la mayor\'ia de los casos, el aumento en complejidad que supone el uso de \textit{JSON-LD} frente a un objeto \textit{JSON} equivalente que no intente representar la serializaci\'on de un grafo \textit{RDF}, se vea \textit{REST}ringido al uso de un par de atributos adicionales.\\

\subsection{Autenticaci\'on y WebID}

Un problema asociado a la construcci\'on de servicios web en general y a la construcci\'on de servicios web sem\'anticos y \textit{APIs} de datos enlazados abiertos, consiste en la autenticaci\'on de los agentes que intentan acceder a la informaci\'on expuesta por el servicio.
Diversos mecanismos de autenticaci\'on han sido propuestos dentro de la comunidad de desarrollo de aplicaciones web, siendo la m\'as popular \textit{OAuth} \cite{hammer2010oauth}.

[TODO]

\subsection{Equivalencia entre el modelo de datos \textit{RDF} y el modelo relacional}

Otras las \'areas en las que la comunidad de Datos Enlazados Abiertos ha desarrollado su actividad es en la de la b\'usqueda de mecanismos que permitan la transformaci\'on de datos expresados en el modelo de datos \textit{RDF} como datos relacionales almacenados en un sistema gestor de bases de datos relacional.\\
Tradicionalmente, la comunidad alrededor de la Web Sem\'antica ha trabajado en el desarrollo de tecnolog\'ias espec\'ificas para el almacenamiento y recuperaci\'on de datos desde un grafo \textit{RDF}, buscando obtener la m\'axima eficiencia posible en el acceso a los datos sem\'anticos.\\
Sin embargo, la tecnolog\'ia m\'as extendida en el mundo del desarrollo web para almacenar la capa de datos de una \textit{API} de servicios web consiste en el uso de un sistema gestor de bases de datos relacional, que permite el acceso a los datos almacenados mediante el lenguaje de consulta \textit{SQL}.\\
La comunidad de Datos Enlazados Abiertos ha intentado encontrar un mecanismo que permita trasladar el modelo de datos \textit{RDF} al modelo relacional de forma que el almacenamiento de datos sem\'anticos \textit{RDF} pueda llevarse a cabo usando bases de datos relacionales, acercando de esta manera el uso de \textit{RDF} a la mayor\'ia de desarrolladores web.
El resultado de este esfuerzo es un borrador de propuesta de est\'andar \textit{W3C} denominada Lenguaje de Mapeado Relacional a \textit{RDF} (\textit{R2RML}) \cite{r2rml}. \textit{R2RML} consiste en una ontolog\'ia gen\'erica que permite establecer una correspondencia entre columnas en un esquema relacional y los componentes de los tripletes de un grafo \textit{RDF}., siendo este mecanismo lo suficientemente potente como para almacenar cualquier grafo \textit{RDF} en las tablas de un sistema gestor de bases de datos relacionales. \textit{R2RML}, tambi\'en incluye soporte para algunas ideas aparecidas dentro de la comunidad de Datos Enlazados Abiertos, como los mencionados grafos con nombre.


\section{Computaci\'on Distribuida}

El consumo de \textit{APIs} de servicios web por parte de agentes software con el fin de llevar a cabo alguna finalidad determinado puede llegar a presentar situaciones de gran complejidad donde diferentes agentes y servicios intercambian peticiones que requieren la coordinaci\'on de las partes involucradas para ser llevadas a cabo con \'exito. Nos encontramos pues ante un escenario donde la computaci\'on se realiza de forma distribuida entre todos los agentes y servicios involucrados en ellas y que puede ser analizada y modelizada usando las herramientas te\'oricas que han sido desarrolladas para el estudio dentro del area de investigaci\'on en computaci\'on distribuida.\\
En las siguientes secciones se detallan algunos de los modelos y desarrollos en este \'area que han sido utilizados para la elaboraci\'on de esta tesis.\\

\subsection{Espacios de tuplas y espacios de tripletes}

Los espacios de tuplas \cite{bussler2005minimal}, es una propuesta de mecanismo de comunicaci\'on entre procesos que realizan alg\'un tipo de computaci\'on distribuida. Un espacio de tuplas consiste en una memoria asociativa que puede ser accedida concurrentemente por diferentes procesos para almacenar, recuperar o eliminar unidades de informaci\'on de composici\'on variable conocidas como tuplas.\\
Cuando dos procesos necesitan coordinarse a trav\'es del espacio de tuplas, el proceso consumidor de informaci\'on intenta realizar una operaci\'on de lectura en el espacio de tuplas para un determinado patr\'on de datos, quedando en un estado bloqueado si ninguna tupla que satisfaga las condiciones impuestas por el patr\'on de lectura se encuentra disponible en ese momento. Cuando el proceso productor desea notificar al proceso consumidor de una condici\'on determinada en la computaci\'on, puede insertar en el espacio de tuplas una nueva tupla de informaci\'on que cumpla con el patr\'on de lectura del proceso consumidor. En el momento en que la tupla es insertada en el espacio de tuplas, el proceso consumidor es desbloqueado y recibe la nueva tupla proveniente del proceso productor que quer\'ia se\~nalar un determinado evento a los procesos consumidores. Este mecanismo, basado en una memoria asociativa usada como mecanismo de coordinaci\'on para los procesos involucrados en una computaci\'on distribuida, se conoce tambi\'en como sistemas de pizarra (\textit{blackboard systems}) \cite{nii1986blackboard}.\\
Se ha demostrado que los sistemas de espacio de tuplas son Turing-completos \cite{busi2000expressiveness}. La implementaci\'on pr\'actica de las ideas expuestas en el modelo de computaci\'on dado por los espacios de tuplas, bajo la denominaci\'on de comunicaci\'on generativa, es Linda \cite{linda}. Las operaciones que sobre el espacio de tuplas ofrec\'ia Linda se puede resumir en la siguiente tabla:\\

\begin{itemize}

\item \textbf{In:} Consume at\'omicamente, extray\'endola de la memoria, una tupla del espacio de tuplas.
\item \textbf{Rd:} Lee at\'omicamente, sin extraerla de la memoria, una tupla del espacio de tuplas.
\item \textbf{Out:} Escribe una nueva tupla en el espacio de tuplas.
\item \textbf{Eval:} Crea un nuevo proceso que interaccionar\'a con el espacio de tuplas usando alguna de las operaciones anteriormente meniconadas.

\end{itemize}

En su concepci\'on original, las tuplas almacenadas en el espacio de tuplas, consisten en estructuras arbitrarias de datos, sin embargo, es posible adaptar el modelo de computaci\'on al uso de datos sem\'anticos, compuestos por tripletes \textit{RDF} con sujeto-predicado-objeto \cite{fensel2004triple}. Desde este punto de vista, un espacio de tuplas se convierte en un espacio de tripletes, que puede ser manipulado por los agentes involucrados en la computaci\'on a trav\'es de las operaciones b\'asicas de manipulaci\'on de espacios de tuplas.

\subsection{C\'alculos de procesos}

Un problema b\'asico dentro del estudio de la computaci\'on distribuida es el del modelado formal de los sistemas concurrentes, de tal forma que los problemas propios de este tipo de computaci\'on, como la bisimulaci\'on entre sistemas,  puedan ser formalizados y analizados mediante una serie de transformaciones algebraicas definidas en dicho c\'alculo.\\
Existe un gran n\'umero de propuestas de diferentes c\'alculos de procesos en la literatura: \textit{CSP} \cite{csp}, \textit{CCS} \cite{ccs}, \textit{Ambient Calculus} \cite{ambient_calculus}, etc. Entre este grupo de formalismos destaca el C\'alculo Pi (Pi-Calculus) \cite{pi_calculus}. En el C\'alculo Pi, una computaci\'on concurrente es llevada a cabo por una red de procesos intercambiando informaci\'on a trav\'es de canales. Los procesos pueden leer de forma bloquenate y escribir a trav\'es de estos canales, de tal forma que pueden ser utilizados como mecanismos de coordinaci\'on. La principal caracter\'istica del C\'alculo Pi es que los mismos canales tambi\'en pueden ser enviados a trav\'es de los canales, de tal forma que los procesos llevando a cabo la computaci\'on pueden ganar acceso a nuevos procesos a trav\'es del intercambio de canales, haciendo que la red de procesos cambie y evolucione en el tiempo.\\

Las principales primitivas presentes en el C\'alculo Pi son las siguientes:\\

\begin{itemize}
\item \textbf{Ejecuci\'on concurrente de procesos:} m\'ultiples procesos est\'an siendo ejecutados de forma simult\'anea en el sistema.
\item \textbf{Comunicaci\'on entre procesos a trav\'es de un canal:} la lectura y escritura bloqueante de datos a trav\'es de un canal permite la coordinaci\'on entre procesos.
\item \textbf{Replicaci\'on:} primitiva que permite la creaci\'on de una nueva instancia de un proceso que se ejecutar\'a de forma concurrente.
\item \textbf{Restricci\'on de nombres:} primitiva que permite a un proceso reservar un nuevo identificador constante que actuar\'a como un nuevo canal por el que transmitir informaci\'on.
\item \textbf{Terminaci\'on:} Permite a un proceso terminar su ejecuci\'on, denot\'andose en el c\'alculo con el uso de un proceso especial 0.
\end{itemize}

Con estas primitivas b\'asicas el C\'alculo Pi constituye un modelo de computaci\'on Turing completo \cite{milner1992functions}.\\
En el mundo de los servicios web \textit{WS-*}, el C\'alculo Pi y los dem\'as formalismos relacionados han sido utilizados reiteradamente como la base te\'orica sobre la que construir mecanismos automatizados para la composici\'on y la orquestaci\'on de servicios web, un ejemplo paradigm\'atico es el intento de usar el C\'alculo Pi para definir la sem\'antica de la propuesta de est\'andar para un lenguaje de orquestaci\'on de servicios web WS-BPEL \cite{lucchi2007pi}.
El tratamiento formal que el uso de un c\'alculo de proceso aporta al estudio de los escenarios complejos de interacci\'on entre agentes y \textit{APIs} de servicios web lo han convertido en una base te\'orica atractiva con la que abordar problemas complejos como la simulaci\'on, la verificaci\'on y detecci\'on de errores en el consumo de servicios web \cite{narayanan2002simulation}, \cite{bhargavan2004verifying}.\\

