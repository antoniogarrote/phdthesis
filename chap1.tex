%% This is an example first chapter. You should put chapter/appendix that you
%% write into a separate file, and add a line \include{yourfilename} to
%% main.tex, where `yourfilename.tex' is the name of the chapter/appendix file.
%% You can process specific files by typing their names in at the 
%% \files=
%% prompt when you run the file main.tex through LaTeX.
\chapter{Introducci\'on}

En el periodo de tiempo transcurrido desde la generalizaci\'on del uso de la World Wide Web, a mediados de la d\'ecada de 1990 hasta el momento actual, 
la riqueza y sofisticaci\'on del contenido disponible para los usuarios de la web se ha incrementado de manera dram\'atica.
El salto de complejidad que media desde la concepci\'on de la web como un conjunto de documentos HTML enlazados entre s\'i por hiperv\'inculos, 
hasta la aparici\'on de las primeras aplicaciones web o la construcci\'on de las muchas redes sociales disponibles hoy en d\'ia, ha supuesto un desaf\'io constante 
para las personas involucradas en el desarrollo de las tecnolog\'ias web que han hecho posible tal evoluci\'on.\\
Lejos de detenerse o incluso frenarse, el concepto de aplicaci\'on web sigue transform\'andose y enfrent\'andose a nuevos desaf\'ios que deben ser solventados
 con el desarrollo de soluciones tecnol\'ogicas que hagan posible ofrecer nuevos servicios capaces de solucionar los problemas de los usuarios actuales de la web.\\
A continuaci\'on enumeraremos algunos de los principales problemas abiertos que encuentran los usuarios de aplicaciones web actuales y que no han sido todav\'ia 
resueltos desde el punto de vista tecnol\'ogico de una forma \'optima:

\begin{itemize}

\item {\bf Agregaci\'on de fuentes de datos:} Un problema com\'un entre los aplicaciones web actuales, especialmente en el \'area de las redes sociales, es el de la fragmentaci\'on de los datos del usuario: fotos, contactos, listas de reproducci\'on musical, etc. entre diversos servicios, de forma tal que el intercambio de estos datos entre estos diferentes servicios es complicado, cu\'ando no imposible, m\'as all\'a de soluciones ad-hoc que no pueden ser reutilizadas entre aplicaciones o intentos de estandarizaci\'on por parte de la industria de mecanismos de autenticaci\'on para un usuario entre los que cabe destcar {\it OAuth} \cite{hammer2010oauth}.

\item {\bf Reconciliaci\'on de recursos:} El problema del intercambio de datos entre diferentes aplicaciones no se limita a la adquisici\'on de estos datos y a la implementaci\'on de la correspondiente capa de privacidad y seguridad en el acceso, sino que tiene una implicaci\'on m\'as fundamental al nivel sem\'antico en cuanto a la denotaci\'on de la naturaleza e identidad de los recursos disponibles en los diferentes servicios. Esto significa que en un entorno abierto y cambiante, donde nuevos servicios y aplicaciones aparecen y desaparecen ofreciendo nuevos tipos de recursos, debe ser posible establecer asociaciones entre estos recursos as\'i como determinar su naturaleza, para poder ofrecer a partir de ese plano sem\'antico, alg\'un tipo de servicio \'util para el usuario final. Esto significa el establecimiento de un vocabulario com\'un y extensible para los recursos web, as\'i como el de un mecanismo igualmente extensible para la designaci\'on, identificaci\'on y recuperaci\'on de dichos recursos entre diferentes aplicaciones.

\item {\bf Serializaci\'on de datos:} Otro importante escollo para la construcci\'on de aplicaciones web distribuidas viene dado por el hecho de que a\'un cuando sea posible conocer la identidad y naturaleza de un recurso disponible en una aplicaci\'on web, y el acceso a dicho recurso venga dado por un mecanismo est\'andar, todav\'ia ser\'ia necesario ofrecer una representaci\'on concreta de dicho recurso para la aplicaci\'on cliente de forma tal que pueda satisfacer la funcionalidad deseada por el usuario. Es por lo tanto necesario ofrecer un mecanismo est\'andar y extensible para decidir el paso del nivel sem\'antico a la representaci\'on sint\'actica del recurso, es decir, su serializaci\'on, o si as\'i se prefiere, el formato concreto en el que se van a consumir los datos asociados del recurso.
\end{itemize}

Estos tres problema combinados afectan a la arquitectura, dise\~no e implementaci\'on de la mayor\'ia de aplicaciones web y servicios sociales en uso hoy en d\'ia como {\it Facebook}, {\it Twitter}, {\it Google+}, etc. lo que supone importantes consecuencias para los usuarios finales de dichos servicios, ya que la complejidad en la interacci\'on entre servicios hace muy dif\'icil la implementaci\'on de nuevas funcionalidades capaces de combinar datos almacenados en distintos servicios con datos ajenos a ellos, ya que esto supone la implementaci\'on de los diferentes mecanismos de acceso, la elecci\'on o implementaci\'on de los componentes software capaces de interpretar las representaciones accesibles y traducir la sem\'antica de los recursos recuperados a una ontolog\'ia com\'un que ser\'a utilizada para implementar la l\'ogica del servicio agregador de datos.\\
La complejidad de este proceso es un obst\'aculo para el usuario a la hora de utilizar los datos que han generado, y de los que es autor, fuera del servicio que us\'o para generar dichos datos y, por \'ultimo, se traduce en una fragmentaci\'on de la identidad de un usuario en la web a trav\'es de una nube de aplicaciones y servicios, que no controla y de la que puede ser privado de forma arbitraria por los propietarios de dichos servicios.\\
Esta concepci\'on de las aplicaciones y servicios web como sistemas cerrados, centralizados y no inter-operables, m\'as all\'a de una forma muy limitada de conectividad, contrasta de forma marcada con la arquitectura de la web: abierta, distribuida y construida sobre est\'andares, que buscan asegurar la inter-operabilidad entre servicios, tal y como ha sido recogida y documentada por sus creadores [cite:principios de arquitectura web] y en principios aceptados ampliamente por la comunidad dedicada al desarrollo web como los principios {\it REST} (Representational State Transfer) \cite{fielding2000representational}.\\

\section{Desarrollo de la iniciativa Web Sem\'antica}
Al mismo tiempo que estos nuevos tipos de aplicaciones y servicios web eran construidos por el grueso de la comunidad de desarrolladores web, la comunidad dedicada a la investigaci\'on sobre Web Sem\'antica, eminentemente acad\'emica, ha ido generando, desde su creaci\'on hace m\'as de diez a\~nos, una serie de ideas que se han traducido m\'as tarde en est\'andares que tienen el potencial para solucionar algunos de de los problemas anteriormente mencionados. Entre estos est\'andares podemos mencionar las siguientes tecnolog\'ias:

\begin{itemize}

\item \textbf{\textit{RDF} \cite{rdf}, un modelo de datos est\'andar y con una sem\'antica formal:} que puede ser utilizado para poder integrar datos entre diferentes aplicaciones y servicios web, usando un grafo donde se establecen relaciones entre recursos usando el mecanismo est\'andar de la web: el hiperv\'inculo.

\item \textbf{\textit{OWL} \cite{owl} un lenguaje extensible para la definici\'on de ontolog\'ias:} definido sobre el modelo de datos propio de {\it RDF}, ofrece un mecanismo eficiente para que diferentes aplicaciones describan el contenido sem\'antico de los datos ofrecidos mediante la construcci\'on de ontolog\'ias o la reutilizaci\'on de ontolog\'ias ya existentes. El uso de {\it RDF} como la base para la descripci\'on de ontolog\'ias {\it OWL} asegura que la recuperaci\'on e integraci\'on de la ontolog\'ia describiendo la sem\'antica de un recurso web en particular s\'olo supone seguir un hiperv\'inculo hasta el documento {\it RDF} que contiene la ontolog\'ia quedando de este modo el recurso integrado en el grafo que describe todas las entidades que ya conoce el agente. {\it OWL} a su vez tiene una sem\'antica formal de Mundo Abierto \cite{owl_semantics} definida estrictamente usando el modelo te\'orico que supone la L\'ogica Descriptiva, lo que permite comprobar la validez y consistencia del resultado de agregar ontolog\'ias provenientes de diferentes fuentes en un s\'olo modelo. 

\item \textbf{Ontolog\'ias \textit{OWL} est\'andar como \textit{Dublin Core} \cite{dublin_core}, \textit{SIOC} \cite{sioc}, etc:} as\'i como otro gran n\'umero de ontolog\'ias cubriendo diferentes dominios de aplicaci\'on, desde el farmac\'eutico \cite{denney2009pharma} hasta la venta en l\'inea \cite{goodrelations}, que pueden ser utilizados directamente por los desarrolladores de aplicaciones para describir su contenido usando un vocabulario com\'un que facilite la inter-operabilidad entre los diferentes servicios web.

\item \textbf{\textit{SPARQL} \cite{sparql} un lenguaje para la recuperaci\'on de meta-datos sem\'anticos:} que permite realizar consultas sobre un grafo {\it RDF} incluyendo entidades provenientes de diferentes servicios y aplicaciones y que permite recuperar la informaci\'on necesaria para que un determinado agente web lleve a cabo su cometido. M\'as a\'un, {\it SPARQL} se puede usar en conjunci\'on con el soporte para la inferencia l\'ogica que provee la sem\'antica formal de {\it OWL} para recuperar informaci\'on adicional inferida y que no necesita ser hecha expl\'icita por las aplicaciones proveedoras de meta-datos.

\end{itemize}

Sin embargo, y a pesar del potencial de est\'as tecnolog\'ias para solucionar algunos de los problemas m\'as importantes de integraci\'on de datos en las aplicaciones web modernas, el uso de dichos est\'andares por parte de la gran mayor\'ia de la comunidad de desarrolladores web ha sido muy limitado, vi\'endose el inter\'es por la Web Sem\'antica restringido a la comunidad acad\'emica y a desarrolladores en dominios de aplicaci\'on muy particulares, como el farmac\'eutico.\\
La principal objeci\'on esgrimida por el grueso de desarrolladores web a la hora de intentar utilizar el conjunto de tecnolog\'ias desarrolladas para la web sem\'antica se centra en la excesiva complejidad de dichas tecnolog\'ias. La base de esta cr\'itica puede estar justificada por la fuerte influencia que la academia ha tenido en el desarrollo de estas tecnolog\'ias, haciendo hincapi\'e en problemas de gran complejidad te\'orica, como la inferencia l\'ogica, en detrimento problemas m\'as sencillos, pero de aplicaci\'on potencial m\'as generalizada o de los detalles concretos de implementaci\'on que son necesarios para la aplicaci\'on pr\'actica de los desarollos te\'oricos realizados, como los formatos para la serializaci\'on de grafos {\it RDF}. 
Como consecuencia de todo esto, alternativas te\'oricamente menos id\'oneas desde el punto de vista formal se han impuesto como las opciones tecnol\'ogicas preferidas para el desarrollo web por la inmensa mayor\'ia de programadores, por ejemplo, el uso de objetos {\it JSON} \cite{json} sin concepto de identidad, no extensible y sin una posibilidad est\'andar para enlazar unos objetos con otros, como el formato de intercambio de datos universal de las aplicaciones web sociales, en vez de cualquier posible serializaci\'on de {\it RDF}.\\

\section{El Enfoque \textit{Open Linked Data}}
Con el objetivo de intentar ofrecer una versi\'on de la Web Sem\'antica m\'as pragm\'atica y f\'acil de utilizar por la mayor\'ia de desarrolladores web surge la iniciativa {\it Open Linked Data} o Datos Enlazados Abiertos. 
Desde este enfoque, la visi\'on de lo que supone la Web Sem\'antica se desplaza de problemas fuertes, como la inferencia l\'ogica sobre los datos, hacia problemas m\'as simples pero fundamentales para el desarrollo web pr\'actico, como el intercambio y la integraci\'on de datos.\\
Para conseguir este fin, la comunidad {\it Open Linked Data} ha adaptado aquellos est\'andares propuestos como parte de la Web Sem\'antica y los ha adaptado a las necesidades del desarrollo web m\'as gen\'erico, por ejemplo, ofreciendo serializaciones de {\it RDF} simples basadas en {\it JSON} \cite{jsonld} o {\it HTML} \cite{rdfa}, con el objetivo de que sean una opci\'on realista para la gran mayor\'ia de desarrolladores web. A su vez, la comunidad {\it Open Linked Data} tambi\'en ha propuesto nuevos est\'andares para solucionar otros problemas b\'asicos en la integraci\'on de datos en aplicaciones web como la autenticaci\'on, como por ejemplo {\it WebID} \cite{webid}.\\
Tanto a la hora de adaptar tecnolog\'ias web sem\'anticas como a la hora de proponer nuevos est\'andares, la comunidad {\it Open Linked Data} siempre ha intentado guiarse en los principios {\it REST} de arquitectura web, ampliamente aceptados por la mayor\'ia de los desarrolladores web, incluyendo pr\'acticas como la negociaci\'on de contenido \cite{holtman1998transparent} o clarificando la distinci\'on entre recursos de informaci\'on y no informaci\'on \cite{fielding2005httprange}. 
Desde el punto de vista Open Linked Data, sus propuestas no son m\'as que la extensi\'on de los principios arquitecturales {\it REST} al intercambio de datos, reutilizando all\'i donde sea posible el trabajo llevado cabo por la comunidad impulsora de la Web Sem\'antica, en lugar de proponer nuevas soluciones desde cero.

\section{Objetivos de esta tesis}

El objetivo que se propone esta tesis sigue la l\'inea de trabajo propuesta por la iniciativa {\it Open Linked Data} pero situando nuestro foco de atenci\'on en la definici\'on y desarrollo de {\it interfaces de programaci\'on de aplicaciones} ({\it APIs}) sem\'anticas.\\
Desde nuestro punto de vista, las {\it APIs} de datos son el componente esencial de las aplicaciones web modernas.\\
La {\it API} de una aplicaci\'on determina los datos y recursos que van a ser accesible por otros agentes web para llevar a cabo sus funcionalidades y, por lo tanto, el grado en que los datos generados por una aplicaci\'on web pueden ser reutilizados e integrados con los datos de otras aplicaciones y servicios. Otros factores que han incrementado la importancia del desarrollo de {\it APIs} de aplicaciones son el enorme crecimiento del mercado de aplicaciones para dispositivos m\'oviles necesitan conectarse con los datos almacenados en el {\it backend} de una aplicaci\'on web a trav\'es de una {\it API} de datos, as\'i como el crecimiento de los clientes pesados {\it JavaScript} en las aplicaciones web de escritorio, que se conectan de la misma manera que un cliente nativo m\'ovil a la {\it API} de datos de la aplicaci\'on.